\documentclass[12pt]{article}
\usepackage{fontspec}
\usepackage[draft]{microtype}
\usepackage{xltxtra}
\usepackage[margin=1in]{geometry}
\usepackage{multicol}
\usepackage{chngcntr}
\usepackage{url}
\usepackage{pdflscape}
\usepackage{setspace}
\usepackage[shortlabels]{enumitem}

\setmainfont[Numbers=OldStyle,Scale=0.9]{TeX Gyre Termes}

% Emulate table numbering per section. Unfortunately we can't let latex
% auto-number the table despite this because the table numbers in the original
% report do not correspond precisely to the section they are in (e.g. Table 9.1
% is in Section 10 and Table 7.1 is in Section 8). So we use counterwithin*
% instead of counterwithin and manually adjust the section ID in affected
% sections to result in the appropriate table ID :(
\counterwithin{table}{section}

\begin{document}

\tableofcontents
\thispagestyle{empty}

\newpage
\pagenumbering{arabic}

\title{Report of the Task Force on Military Software}
\author{Defense Science Board}
\date{September 1987}

\maketitle

\emph{NOTE: This is a modified version of the original report. See note below.}
\footnote{This report is an adapted version of the original report from
\url{https://apps.dtic.mil/dtic/tr/fulltext/u2/a188561.pdf} which has been
semi-automatically converted to a computer-readable form for ease of access and
reference in this modern Internet-based ecosystem. The conversion was performed
by Michael Pyne. He worked with some level of care but may not have caught all
transcription errors.}

\section{Executive Summary}

Many previous studies have provided an abundance of valid conclusions and
detailed recommendations. Most remain unimplemented. If the military software
problem is real, it is not perceived as urgent. We do not attempt to prove
that it is; we do recommend how to attack it if one wants to.

We do not see any single technological development in the next decade that
promises ten-fold improvement in software productivity, reliability, and
timeliness. There are several technical developments under way which together
can be expected to yield one order of magnitude, but not two. Few fields have
so large a gap between best current practice and average current practice; we
concur with the priorities that DoD has given tc upgrading average practice by
more vigorous technology transfer.

Current DoD initiatives in software technology and methodology include the Ada
effort, the STARS program, DARPA’s Strategic Computing Initiative, the
Software Engineering Institute, and a planned program in the Strategic Defense
Initiative. These five initiatives are uncoordinated. We recommend that the
Undersecretary of Defense (Acquisition) establish a formal program
coordination mechanism for them, (Rec. \#2).

The big problems are not technical. In spite of the substantial technical
development needed in requirements-setting, metrics and measures, tools, etc.,
the Task Force is convinced that today’s major problems with military software
development are not technical problems, but management problems. Hence we call
for no new initiatives in the development of the technology, some modest shift
of focus in the technology efforts under way, but major re-examination and
change of attitudes, policies, and practices concerning software acquisition.

\addcontentsline{toc}{subsection}{STARS}
\subsection*{STARS}

The DoD program for Software Technology for Adaptable, Reliable Systems,
STARS, has made little progress in recent years and has had vague and
ill-focused plans for the future. Service support and enthusiasm is lacking.
Yet it is very important that such a project-independent methodology
development effort proceed. We recommend that the STARS Joint Program Office
be moved from the Office of the Secretary of Defense to the USAF Electronic
Systems Division. (Rec. \#1) We recommend that a general officer be given
responsibility for STARS, the Ada Joint Program Office, and Software
Engineering Tnstitute (whose contracting office is already in ESD). Deputies
from the other Services should be appointed.

\addcontentsline{toc}{subsection}{Ada}
\subsection*{Ada}

It is very important for DoD to have a standard programming language; Ada is
by far the strongest candidate in sight. The 1983 mandate for Ada was
technically premature. DoD commitment to Ada since that time has been weak.
The state of Ada compiling technology is now such that it is time to commit
vigorously and wholeheartedly.  The directives 3405.1 and 3405.2 are right
first steps - management follow-through on enforcement and support is now
essential.

Ada embodies and facilitates a set of new approaches to building software,
generally known as “modern software practices.” We expect these practices,
rather than yet another programming language, to make a real difference in
software robustness, reusability, adaptability, and maintenance. Ada is not
the only conceivtble vehicle for such practices, but if is here, it has been
tailored for the embedded software problem, and multiple compilers have- been
validated. We recommend against further waiting, language tuning, or
subsetting.

Achieving the benefits of modern programming practices requires the development of
unified programming environments. This work must continue to be pushed forward.
Few program managers will want to take on the headaches of being first user of a new
tool, yet it is essential that all major new programs be committed to that tool if it is to
be effective. Only top-level DoD commitment and mandate can make that happen.
We commend AJPO for its technical success in establishing the language definition and
language validation procedures. We recommend that it be moved from OSD to a unified
software joint program office in the USAF Electronic Systems Command (Recs. \#6,7).

\addcontentsline{toc}{subsection}{Acquisition}
\subsection*{Acquisition}

\paragraph{Mileu.} The civilian software market has exploded in the past decade, so that the total
civilian market for purchased software, not counting in-house-built application software, is
now more than ten times larger than the DoD market. This requires a radical update in
much DoD thinking. Some implications:

\begin{enumerate}
\item DoD can no longer create de facto standards and enforce them on the civilian market,
as it was able to do with COBOL.

\item DoD must not diverge too far from whatever the civilian market is doing in programming methodology, else it will have to support its own methodology by itself, with little
resource or training commitment from others. (The same thing is true of processor
architectures.)

\item DoD should be aggressively looking for opportunities to buy, in the civilian market,
tools, methods, environments, and application software. Whenever it can use these
instead of custom-built software, it gets big gains in timeliness, cost, reliability,
completeness of documentation, and training. But today’s acquisition regulations
and procedures are all heavily biased in favor of developing custom-built software
for individual programs.
\end{enumerate}

\paragraph{Life-cycle model.} DoD Directive 5000.29 and STD 2167 codify the
best 1975 thinking about software, including a so-called “waterfall” model
calling for formal specification, then request for bids, then contracting,
delivery, installation, and maintenance. In the decade since the waterfall
model was developed, our discipline has come to recognize that setting the
requirements is the most difficult and crucial part of the software building
process, and one that requires iteration between the designers and users. In
best modern practice, the early specification is embodied in a prototype, which
the intended users can themselves drive in order to see the consequences of
their imaginings. Then, as the design effort begins to yield data on the cost
and schedule consequences of particular specifications, the designers and the
users revise the specifications.

Directive 5000.29 not only does not encourage this best modern practice, it
essentially forbids it. We recommend that it be revised immediately to mandate
and facilitate early prototyping before the baseline specifications are
established (Rec. \#23).

DoD-STD-2167 likewise needs a radical overhaul to reflect best modern practice.
Draft DoD-STD-2167A is a step, but it does not go nearly far enough. As
drafted, it continues to reinforce exactly the document-driven,
specify-then-build approach that lies at the heart of so many DoD software
problems.

For major new software builds, we recommend that competitive level-of-effort
contracts be routinely let for determining specificatlons and preparing an
early prototype (Rec. \#26). The work of specification is so crucial, and yet
its fraction of total cost is so small, that we believe duplication in this
phase will save money in total program cost, and surely save time. After a
converged-specification has been prepared and validated by prototyping, a
single competitively-bid contract for construction is appropriate.

\paragraph{Incentives.} Defense procurement procedures discourage contractor
investment in the development of new software methodology. Any such contractor
investment made today promises low return. We recommend that the DoD
rights-in-data policy be revised to distinguish software rights from other
rights, and that the policy as it applies to software be designed to encourage
contractor investment, both with private and IRD funds, in tools, methods, and
programming environments (Recs. \#17-22).

Similarly, today’s policies actively discourage the reuse of software modules
from one system in another. We recommend a variety of policy changes, each
designed to encourage reuse, and indeed, the establishment of a public market
in reusable software parts (Recs. \#29-33).

\addcontentsline{toc}{subsection}{Personnel}
\subsection*{Personnel}

It appears that the number of software-qualified military officers has been
essentially constant over the past decade, despite exponential growth in
software. Many studies have recommended actions that need to be taken re
training, specialty codes, career paths, etc., to address the shortage of
uniformed specialists. Some of these have been taken.  Nevertheless, the number
has not increased.

We doubt that it will. The powerful civilian demand for such persons will, we
expect, continue to drain them away from the Services as fast as they reach
first retirement age, or before. % the PDF is missing the word prior to
                                 % 'before', I believe it is 'or'

Therefore we recommend that the Services now assume that there will not be more
such people, and concentrate effort on how best to use those they have (Rec.
\#34). The application-knowledgeable, technically skilled leaders are the
military’s limiting resource in using today’s computer technology.

We observe that in the best military software programs, the number of customer
software people engaged in the acquisition and program oversight approximates
10\% of the number of contractor personnel. This number does not seem too high.
Few program offices are staffed so well, however, largely due to the shortage
of qualified people. Meanwhile one observes some substantial software-building
efforts under way within the Services, usually done by a combination of
civilian and uniformed personnel, generally managed by software-qualified
officers. This is a second-best use of the available specialist officers.

We recommend phasing out this practice and concentrating the available
knowledgeable officers on acquisition (Rec. \#35). We see no other way that the
exponential growth in needed military software can be met.

\section{Introduction}

\subsection{The Charge to the Task Force}

\subsubsection*{Abbreviated Terms of Reference, (Appendix A2 contains the full text)}

\begin{enumerate}[A.]
    \item Assess and unify various recent studies.
    \item Examine why software costs are high.
    \item Assess STARS for military software; discuss the priority of its components.
    \item Recommend how to enlist industry, Service, and university efforts in a productivity thrust.
    \item Assess STARS, etc., for U.S. international competitiveness.
    \item Recommend how to apply R\&D funds to get the most increase in military software capability.
    \item Recommend how to implement an incremental and evolutionary approach to (F).
    \item Assess the wisdom of the Ada plan, especially in view of “Fourth-Generation” languages.
\end{enumerate}

\subsubsection*{What the Task Force Did Not Address}

\paragraph{Problem Seriousness Sizing.} It would be presumptuous, and appear to
be self-serving as well, for this Task Force to tell the Service commanders and
the DoD civilian authorities that your mission-critical software problem ranks
high on your present or future critical problem list. Other studies have sized
the cost and recounted software-caused delays and system malfunctions. Your own
experience will have to put this problem into proper perspective among all your
difficulties.

What the Task Force is qualified to do for you is to
\begin{itemize}
    \item characterize software, its problems, and its technology

    \item identify trends that will, in the course of time, make today’s
        problems worse or better,

    \item suggest actions to address today’s problems and avert tomorrow’s
        calamities.
\end{itemize}

\paragraph{Non-Mission-Critical Software.} The Task Force largely limited
itself to mission critical systems, those wherein military software most
differs from civilian-market software.  Our recommendations with respect to
procurement, however, apply to all DoD acquisition of software. In Section
\ref{sec:civ-market} we categorize DoD software according to the degree to
which it must be non-standard because of its military function.

\paragraph{Service-Specific Personnel Problems.} We did not address
Service-specific personnel and skills problems. These have been adequately
addressed in earlier studies. The career path and skills-retention problems
continue to be very real in all the Services.

\paragraph{SEI.} We did not review the Software Engineering Institute, other
than to hear a briefing on its objectives. It was in the process of being
established and finding a permanent director during our study; any review would
have been premature.

\paragraph{SDI.} The same was true of the SDI plan for developing software
methodology. At the time we were briefed by the SDI office, there was no plan
to review.

\paragraph{SCI.} We had only one briefing on the DARPA software methodology
efforts encompassed within the Strategic Computing Initiative. These efforts
are properly aimed at producing results a decade hence. The approaches are
sufficiently bold that little in the way of directly applicable short- and
mid-term results can be expected.

\paragraph{New Technological Initiatives.} We do not recommend any new
initiatives or funding for new specific research or technology-development
programs. We support the recent technological initiatives, but today’s major
unaddressed problems are not technical, but managerial.

\subsection{Military Software}

\paragraph{Role.} Software plays a major role in today’s weapon systems. The
“smarts” of smart weapons are provided by software. Software is crucial to
intelligence, communications, command, and control. Software enables
computerized systemns for logistics, personnel, and finance. The chief
“military software problem” is that we cannot get enough of it, soon enough,
reliable enough, and cheap enough to meet the demands of weapon systems
designers and users. Software provides a major component of U.S. war-fighting
capability.

\paragraph{Growth.} DoD software-intensive systems have grown exponentially,
reaching an annual software expenditure level in mission-critical computer
systems of about \$9 billion in 1985, with projections of \$30 billion annually
by 1990 [Taft, 1985].  This continuing growth has strained the ability of the
DoD to manage their development. Because software controls system function,
deficiencies in software development affect over-all weapon system performance
and cost quite out of proportion to the software cost itself.

\paragraph{Like Civilian Software.} Military software is fundamentally like
advanced civilian software, only more so. That is, the properties of real-time
operational software in civilian banking, airline reservations, or process
control, are the same as those of weapon-system software. Big civilian
database and file systems look essentially like the military logistics,
finance, and personnel software. In the operation of a ship or a base, one
finds many small computers whose tasks are essentially the same as those in
civilian businesses.

\paragraph{Only More So.} Mission-critical military software is more
universally real-time, communications-oriented, and resource-constrained than
its civilian counterparts. At any given time, the demands of weapon systems
stress the state of the software art more severely than do civilian demands.

\paragraph{Timeliness and Reliability.} Although the cost of military software
is commonly seen as the major problem, and is emphasized in our Terms of
Reference, both previous studies and our briefers suggest that software
timeliness and reliability are even more critical problems today.

Software development cycles are long, relatively unpredictable, and come at the
end of total weapon system development. Thus they frequently encounter delays,
delays usually on the critical path to operational capability. It also takes
too long to adapt running software to changing hardware or operational
requirements.

Software reliability is equally of concern. Since operational software is
complex, it usually contains design flaws, and these are hard to find and often
painful in effect.

\paragraph{Requirements-Setting Is The Hardest Part.} As is true for complex
hardware systems, the hardest part of the software task is the setting of the
exact requirements, including numbers for size and performance, and including
the relative priorities of different requirements in the designers’ inevitable
trade-offs.

We have no technology and only poor methodologies for establishing such
requirements. There are not even good ways in common use for even
\emph{stating} detailed requirements and trade-off priorities. Misjudgements in
requirements badly hurt effectiveness, cost, and schedule. Such misjudgements
abound. Most common is the specification of over-rich function, whose bad
effects on size and performance become evident only late in the design cycle.
Another common error is the mis-imagination of how user interfaces should work.

In our view the difficulty is fundamental. We believe that users \emph{cannot},
with any amount of effort and wisdom, accurately describe the operational
requirements for a substantial software system without testing by real
operators in an operational environment, and iteration on the specification.
The systems built today age just too complex for the mind of man to foresee all
the ramifications purely by the exercise of the analytic imagination.

This inherent difficulty is unnecessarily compounded in DoD by the presence of
too many intermediaries between the ultimate user and the software specifier.

\paragraph{The Big Problems Are Not Technical.} In spite of the substantial
technical development needed in requirements-setting, metrics and measures,
tools, etc., the Task Force is convinced that today’s major problems with
military software development are not technical problems, but management
problems. Hence we call for no new initiatives in the development of the
technology, some modest shift of focus in the technology efforts under way, but
major re-examination and change of attitudes, policies, and practices
concerning software acquisition.

\subsection{Why Is Software Technology Developing So Slowly?}

Participants and observers in the computer game often marvel that the software
technology develops so slowly, especially in comparison with computer hardware
technology.  In our Terms of Reference we are charged with examining the
underlying nature of the software process so as to explain high costs and slow
development.

\subsubsection*{Hardware Technology Is So Fast.}

The remarkable fact is not the slow rate of development of computer software
technology, but the fast rate of hardware technology, a fact especially
striking to those of us who do both. Today’s hardware offers at least a
10,000-fold gain in price-performance over that of 30 years ago, and one can
choose at least 1000-fold of that gain in either price or performance! No other
technology has come even close to that rate of development. It reflects the
shift of computer hardware from an assembly technology to a process technology.

\subsubsection*{Software Is Labor-Intensive.}

Software development is and always will be a labor-intensive technology. The
work and the time is all in development, not production. Development is always
labor-intensive.  Moreover, in the ultimate, one is developing conceptual
structures, and although our machines can do the dog-work and can help us keep
track of our edifices, concept development is the quintessentially human
activity.

\subsubsection*{The Essence Is Designing Intricate Conceptual Structures Rigorously.}

In Appendix A5, we analyze the software task. We argue that its essence is the
designing of intricate conceptual structures, rigorously and correctly. The
part of software development that will not go away is the crafting of these
conceptual structures; the part that can go away is the labor of expressing
them. The task is made more difficult by three other properties of software
products: (1) the necessity for them to conform to complex environmental,
hardware, and user interfaces; (2) the necessity for them to change as their
interfaces change; (3) and the invisibility of the structures themselves.

We believe a significant fraction of software development effort today is
expended on this essential labor, rather than on the task of expressing the
designs.

\subsubsection*{The Removal of Expression Difficulties Has Brought Much of the Past Gain.}

The essential labor itself has not always taken most of the effort. Much of the
work was formerly spent on non-essential, incidental difficulties in the
expression of the conceptual structures. The three big breakthroughs in
software methodology each have consisted of removing one of these incidental
difficulties.

First was the awkwardness of machine language. High-level languages removed
this difficulty and improved productivity ten-fold.

Second was the loss of mental continuity occasioned by slow turn-around batch
compilation and execution. Time-sharing removed this difficulty, improving
productivity 2-5 times.

Third was the utter incompatibility of files, formats, and interfaces among
various software tools. Integrated programming environments such as Unix and
Interlisp overcame this difficulty, again doubling (or better) productivity.

\subsubsection*{What’s In the Cards?}

There are stll non-essential expression difficulties, but they do not account
for most of the development effort in modern software shops. Future
methodological imprcvements will have to attack the essence - conceptual design
itself.

Examination of the most promising technological developments shows no single
technique that can be expected to yield as much as a 10-fold improvement in
productivity, timeliness, and robustness in the next ten years.

On the other hand, all of the various technological developments on the horizon
together should easily yield a 10-fold improvement in the next decade. It is
not likely that all those developments together will yield a 100-fold
improvement.

\subsection{Current Software Trends}

Five developments in the past decade have revolutionized the software scene.
DoD software practices evolved in the 60s and 70s, and they neither take
into account nor utilize these advances.

\subsubsection*{The Microcomputer Revolution and the Personal Computer}

The microcomputer, both as a component, and by its incorporation into personal
computers, has totally changed the computer field and the software field. Every
procedure for computer acquisition, etc., must now define a floor in machine
size below which it is not applicable, and machines below the floor should be
treated as commodities, components, and spare parts. (Not all procedures have
yet been so revised.)

Obviously software standards such as the Ada mandate must have such a floor as
well. The constraints on embedded microprocessors are such that their
software often must be in machine language. We do not address microprocessor
software.

We likewise do not deal with the software problems of personal computers. DoD,
like every large enterprise, needs some standards as to how such machines are
to be supplied, how they will be equipped with standard-function programs, and
how they are to interchange information. Such standardization should be
minimal and light-handed. We should not recommend that the Ada mandate cover
personal computers. % the word 'should' is hard to read but I think I see the 's'

America’s greatest comparative military advantage is the individual initiative
and ingenuity of our Service people. We are therefore greatly encouraged to see
the Services making personal computers readily available to individual units so
that individuals can solve their own simple computing problems their own way. A
personal computer and an electronic spread sheet make a powerful combination,
sufficient for countless tasks.\footnote{Jones and Brooks had the opportunity
to observe a Blue-Flag simulated Air Force-Army-Marine tactical exercise. We
saw a number and variety of personal computers that have been integrated
effectively into unit operational functions; we were pleased to see a light
dependence on massive computer systems.}

\subsubsection*{A Mass Market for Software}

The personal computer revolution has explosively fueled the development of a
mass market for third-party developed software. This is the most important
development in the software field in our time.

Each of several computer architectures (the properties of a computer that
determine what programs it will run) define a market. The biggest are those for
IBM PC-compatibles, Apple-compatibles, Macintosh, DEC VAX-compatibles,
Unix-compatibles, and IBM 370 compatibles. For each of these markets literally
hundreds of packages are available, covering an immense spectrum of functions
and costing from a few dollars to a few hundred thousand. The markets are
fiercely competitive.

\subsubsection*{Technology for Software Modularization and Reuse}

Techniques for designing software in little modules, for defining the module
interfaces precisely, and for using common file formats have come into standard
use during the decade. These methods, the backbone of so-called “modern
programming practices”, radically improve the structure and adaptability of
large programs. They also define modules, whose reuse often costs one-tenth as
much as writing another module to do the same function. Reuse is also much
quicker, and it yields better tested, more reliable code.

The Ada programming language is designed to make such modularization natural,
and to provide very powerful facilities- for linking modules. Integrated
programming environments, such as Unix, provide the same kind of facility at
another level, that of the shell-script linking whole programs together.

\subsubsection*{Rapid Prototyping and Iterative Development}

As people have recognized that the requirements, and especially the user
interface, require iterative development, with interspersed testing by users,
there has developed a technology for constructing “rapid” prototypes. Such a
prototype typically executes the main-line function of its type, but not the
countless exceptions that make programming costly. It usually does not have
complete error-handling, restart, or help facilities. The prototype is often
built using a lash-up of handy components that swap performance for rapid
interconnect ability. It is usually run on a computer that is bigger and faster
than the target machine.

Commercial packages enable one to prototype graphics interfaces, for example,
so that user testing can be done quite early in the development.

\subsubsection*{Professional Humility and Evolutionary Development}

Experience with confidently specifying and painfully building mammoths has
shown it to be simplest, safest, and even fastest to develop a complex software
system by building a mninimaal version, putting it into actual use, and then
adding function, enhancing speed, reducing size, etc., according to the
priorities that emerge from the actual use. Software engineers must recognize
that we cannot specify mam~moths right the first time. In practice, Version 2
is usually under development before Version 1 is delivered, so Version 3 may be
the first to be affected by actual experience.

This procedure speeds first delivery. It also provides for the iterative
setting of requirements. It minimizes the specification of heavy function whose
performance penalties have not yet been weighed. It tends to concentrate
development effort where it will make the most difference. Seeing the minimal
version run does wonders for the morale of the development team and
substantially boosts their communication as to further development.

Evolutionary development is best technically, and it saves time and money. It
plays havoc with the customary forms of competitive procurement, however, and
they with it. Creativity in acquisition is now needed.

\subsection{Current DoD Programs on Software Technology}

Besides some substantial efforts in individual Service laboratories, DoD has
under way five programs aimed at enhancing software methodology:

\paragraph{STARS.} The program for Software Technology for Adaptable, Reliable
Systems, managed by the Undersecretary of Defense (Acquisition), was started in
ii80 to address all aspects of modern software methodology as applied
especially to mission-critical computer systems.

\paragraph{ADA.} The Ada program, managed by the Ada Joint Program Office under
the Under\-secretary of Defense (Acquisition), was started in 1975 to define
and develop a standard high-level language suitable for embedded computer
systems.

\paragraph{Software Engineering Institute.} Founded in 1964, the SEI mission is
focused on technology transfer - bringing the best modern methodology into
actual practice in the Services and among DoD contractors. The SEI is operated
by Carnegie-Mellon University under contract from the USAF Electronic Systems
Command.

\paragraph{Strategic Computing Initiative.} A software component under the
DARPA Strategic Computing Initiative is aimed at developing radically new
methods and tools, especially those based on expert systems and other
artificial intelligence techniques. The program aims at results a decade ahead
of modern practice.

\paragraph{Strategic Defense Initiative.} A software component under the
Strategic Defense Initiative is aimed at providing methodological advances for
the building of the massive distributed, ultra-high-performance software system
demanded by the SDI.

\subsection{Recent Previous Studies}

The Task Force reviewed the available studies done since 1982, starting with
the monumental 1982 Druffel study, which in turn summarized the results of 26
previous studies. Appendix A4 lists the recent ones.

To a surprising degree, the conclusions of these studies agree with each other
and remain valid; the recommendations continue to be wise. The chairmen of the
several study groups briefed us. All had one message: very little action, has
been taken to implement the recommendations. If the military software problem
is real, it is not perceived as urgent by most high military officers and DoD
civilian officials. Our Task Force does not undertake to prove that it is
urgent; we do tell how to attack it if one wants to.

\section{STARS -- Software Technology for Adaptable, Reliable Systems}

The STARS program objective is to achieve by 1995 a dramatic improvement in our
ability to build reliable, cost-effective defense software by applying known
new technology.  STARS seeks improvement in methods, techniques, tools,
personnel practices, and business practices.

The Task Force examined the STARS program on several occasions during the past
two years. The program is floundering. Little has been achieved during the last
several years. OSD management has recognized the problems, and some remedial
steps are under way. It is too early to tell if these steps will work.

\subsection*{Findings}

\subsubsection*{STARS as originally formulated is a very good idea.}

Members of the Task Force do not expect to see dramatic near-term research
discoveries. However, many incremental improvements in software engineering
have been made over the last decade, and will continue to be made. These
advances could improve our war-fighting capability if they were practiced in
DoD programs now. STARS can accelerate their application.

\subsubsection*{OSD has not provided the vital leadership needed; until recently
STARS has lacked a director with strong technical and management ability.}

The program had no permanent program manager for over a year. Consequently, it
lacked leadership, guidance, and vitality. There has been no single vision of
program objectives, no coordination of spending, no monitor to ensure that
components of the program were complementary, and no assurance that the program
acted in response to the software problems of the Services. Strong top-down
leadership, both technical and administrative, is required. A new, permanent
director has recently been appointed.

\subsubsection*{The program plan has been fuzzy.}

The Task Force had difficulty in identifying specific goals or plans to achieve
them. The program plan does not even recognize the existence of some major
software trends, such as the personal computer revolution. STARS should enable
solutions, not develop them from first principles. It should identify possible
technical approaches and tools, harness the marketplace capability to produce
solutions, and ensure early application to real mission-critical system
developments.

The STARS Program as it stands today has become focussed at a particular,
well-defined part of the military software problem—custom systems, new or
converted, middle-sized to large, whether embedded or command and control.
STARS addresses follow-through software engineering support for ADA software.

This focusing is entirely commendable and beneficial; indeed, it was badly
needed. A corollary is that the problems STARS does not address have also
become clear. They include:

\begin{itemize}
    \item personal computers, workstations, and little software systems built on them
    \item acquisition of off-the-shelf commercial software
    \item supercomputer calculations (still and perhaps forever in FORTRAN)
    \item old systems not worth converting
\end{itemize}

To enumerate these is not to criticize the STARS program. It never really could
address all the problem; it only claimed to do so as long as it was fuzzy and
unfocussed.

\subsubsection*{Balance between program elements has been missing.}

Devising a single software engineering environment dominates the attention of
the program. In contrast, emphasis on multiple possible environments, even some
that are off-the-shelf, would serve the objectives better. Most of what the
STARS program proposes to deliver is scheduled very late in its lifetime. Early
operational milestones would better speed transfer of the technology to the DoD
and civilian practitioners.

\subsubsection*{The program is organized as uncoordinated activities; many are executed
by part-time volunteers.}

An independent committee explores each activity area; little communication
relates the committee actions. For example, there is insufficient integration
of the activities of the business practices area with each of the technical
areas.

\subsubsection*{STARS needs better coordination with the Services, the Software
Engineering Institute, AJPO, DARPA’s Strategic Computing Program and the
Strategic Defense Initiative.}

All these programs have interlocking interests and development programs. Links
have not been carefully established for the input of Service needs into STARS
planning and the output of STARS back to the Services.

It is difficult to get the needed funds allocated for software engineering; if
STARS is terminated a major opportunity will be lost.

An effective STARS Program is indeed needed to accelerate the application of
the best ideas from the laboratory to weapon systems development. If an
effective STARS Program does not materialize, software risks will remain high.

\subsubsection*{Salvage of STARS may not be possible, but it should be attempted. Drastic
action is required.}

\subsection*{Recommendations}

\subsubsection*{\textit{Recommendation 1: Move STARS and rebuild it.}}

Create a Joint Program Office to oversee the STARS program, AJPO, and the
Software Engineering Institute. This Office should be headed by a flag-rank
military officer in order to demonstrate DoD cammitment to provide firm
oversight, resources, and control over DoD software technology efforts. This
Joint Program Office should report to the Deputy Undersecretary of Defense for
Research and Advanced Technology. Locate it at the Electronic Systems Division
in Bedford, Massachusetts, with the Air Force as executive agent.

This management organization has been an effective technique for mustering
Service cooperation on joint efforts in the past. Examples include WIS, Joint
Tactical Fusion, JTIDS, and the Joint Surveillance and Target Acquisition Radar
System. OSD retains oversight authority, and the Joint Program Office
organization will ensure that benefits accrue across all the Services.

\subsubsection*{\textit{Recommendation 2: Task the STARS Office, the Ada Joint Program Office,
the Software Engineering Institute, the SDI software methodology program
element, and the DARPA Strategic Computing Program to produce a one-time
joint plan to demonstrate a coordinated DoD Software Technology Program.}}

This plan must ensure ongoing technical exchange among the five programs.

\subsubsection*{\textit{Recommendation 3: Task the new STARS Director to define a new set of
program goals together with an Implementation plan; emphasis should be on
visible, early milestones that have demonstrable results.}}

This plan should emphasize widespread adoption of the best that exists today.
It should provide incremental products. It should complement what the
commercial sector is doing and focus on DoD-unique requiremrents. It should be
realistic.

\subsubsection*{\textit{Recommendation 4: Direct STARS to choose several real
programs early in development and augment their funding to ensure the use of
existing modern practices and tools.}}

\section{ADA}

DoD defined the Ada language (see MIL-STD-1815A) to be its common,
machine independent programming language for DoD-wide use in mission-critical
computer applications. This intent was established in 1981 by draft versions of
DoD Instruction 5000.31. A subsequent June 1983 memorandum from Dr. Richard
DeLauer, Undersecretary of Defense (R\&E), mandated the use of Ada on all new
DoD mission-critical computer procurements entering concept definition after 1
January 1984 or entering full-scale development after 1 July 1984. Mr. Don
Hicks, Undersecretary of Defense (R\&E), reaffirmed thie mandate to Ada in
December, 1985, as did Secretary Weinberger in November, 1986.

The Task Force discussed Ada, its compilers, and its application in military
programs with the three Services, inter-Service programs such as WIS, and the
acting Director of the Ada Joint Projects Office. The Task Force also
considered fourth-generation languages and their implication for the Ada
effort.

\subsection*{Findings}

\subsubsection*{\textit{Improved Software Engineering Techniques}}

\subsubsection*{Software engineering methods and techniques have dramatically
advanced over the last decade, yet these techniques are not generally practiced
in DoD.}

Ada is not merely a programming language; it is a vehicle for new software
practices and methods for specification, program structuring, development and
maintenance. Without enforced usage of such a vehicle, the radical improvements
in software engineering will not move rapidly into use. Standardization on a
language is the best way to introduce the new practices rapidly.

\subsubsection*{\textit{Ada, The Standard Language}}

\subsubsection*{It is a major technology step forward for the DoD to insist
that all software be built in a high-level language. It is a major management
step forward to standardize on a single high-level language.}

It is not simple to do so; Fortran and Cobol will each survive in some military
applications. The driving reasons to standardize new development in one
high-level language remain valid. Specifically, the quality of the resulting
software will be higher. Enhancement of function, adaptation to environmental
changes, and fixing of errors will be less buggy and cheaper.

Even where exceptions to the use of Ada are granted, all software can and
should be designed using Ada as a design language.

\subsubsection*{Ada was designed by the DoD to be that standard language; It is
the best candidate for standardization available today; it promises to remain
so for the foreseeable future.}

Ada’s constructs support modern software technology and discipline. Ada
supports the evolution and maintenance of reusable software, portable software,
and real-time software. The language definition is precise enough.  Other
candidate languages have many more deficiencies than Ada with respect to the
DoD’s needs.

\subsubsection*{Ada is admittedly complex. This complexity has contributed
significantly to the slow maturation of the language and of its compilers and
tools.}

Enough Ada compilers now exist to demonstrate they can be built. Because of
language complexity, current compilers execute slowly in comparison to a good
Fortran compiler. However, the compilers are doing more checking, and pointing
out errors to the programmer; this is cost-effective. Engineering refinement of
compilers will yield acceptable, even good, Ada compiler speed in the near
term. Moreover, modern partial compilation techniques today reduce the impact
of raw compiler speeds.

\subsubsection*{Due to Ada’s complexity, the code generated by current Ada
compilers Is not yet highly optimized.}

Again, engineering refinement will produce optimizing compilers in the near
term. Whereas Ada application code can be quite slow if all dynamic checking
is enabled, most checking can be turned off in the production version of
application code. There is no technical obstacle to achieving optimized code
for applications written in Ada.

\subsubsection*{The DeLauer mandate to use Ada was premature; it could not be
followed In 1983 because of slow maturation of the language and its
compilers.}

Consequently it became toothless. The compilers have been developed to a point
that the mandate can be implemented now; it should be.

\subsubsection*{Switching to Ada necessitates an up-front investment In order
to reap longer term benefits.}

One cost is education. Teaching Ada also implies teaching the new software
engineering practices and disciplines. This must be done anyway. Forcing this
learning is a major motive for adopting Ada quickly and extensively.

Computer time costs will be somewhat higher because of the slower compiling.
These costs are transient and will go down as Ada programming environments are
widely installed, as the software tools mature, and as hardware
cost/performance continues to drop.

\subsubsection*{Although incurring the up-front costs is wise for DoD,
individual program managers and contractors have no incentive to do so.}

The costs of training, compiler and tool acquisition, and running the current
immature compilers are present and readily measured, whereas the benefit is
future and more difficult to measure. Adoption must therefore be mandated by
high management.

\subsubsection*{Ada is being successfully used today in military programs, such
as AFATDS.}

At least sixty-four validated compilers exist, with more in the wing. Moreover,
Ada is not just a DoD captive language. Civilian commitment to Ada is emerging.
It is noteworthy that the majority of compilers for Ada are built with private,
not DoD, funds.  One cannot predict, however, that Ada will become the standard
language for civilian data proceasing, as Cobol did. Too many different forces
are at work.

\subsubsection*{\textit{Fourth-Generation Languages}}

\subsubsection*{Fourth-generation languages are application-specific program
generators; because they are not general purpose, they are not in competiton
with Ada.}

The term fourth-generation is a misnomer. It has been used to characterize a
wide variety of languages which are not descendants of the third-generation,
general-purpose languages. The term encompasses application-specific languages
such as database languages and electronic spread sheets, program generators,
non-procedural languages, and even artificial intelligence languages such as
Prolog. Each language is designed to be applied to problems in a limited
domain. Therefore the fourth-generation languages do not compete with Ada.

\subsubsection*{If an application is well-matched to a fourth-generation
language, the cost of realizing the application can be a hundredfold less
expensive than programming it in any general purpose language, Including Ada.}

Spreadsheets are routinely used to accomplish tasks in minutes that would
require hours of work in a general purpose language. Similarly, an exploratory
artificial intelligence (AI) task may be programmed in an AI language in days
versus months in any general purpose language.

\subsubsection*{A weapons system development is not one task in a single
problem domain; the Task Force is skeptical that any fourth-generation language
is well-suited to such applications.}

Note that some of the high-risk tasks in a weapons system may be advantageously
prototyped in a fourth-generation language to experiment with algorithms or
software structure before actual development commences. Similarly, some
single-domain mission critical applications, such as some intelligence data
processing, may be cost-effectively implemented in a fourth-generation language
such as a database provides.

Some efforts to develop large software systems entirely in a fourth-generation
language, such as the New Jersey Motor Vehicle Registration System, have been
unsuccessful.

\subsubsection*{\textit{DoD Management of Ada}}

\subsubsection*{Only top DoD management can sustain a policy and program for
incurring the costs and risk of early DoD use of Ada.}

Contractors incurring the up-front costs must have assurance that investment in
Ada tooling will pay off. Programs must plan for long-range cost and quality
improvements.

\subsubsection*{The Ada Joint Program Office (AJPO) is the DoD’s focal point
for policy and coordination of Ada standardization, validation, and language
control; it has done a commendable job in achieving its technical objectives.}

The AJPO has maintained a stable language definition. It has defined a
comprehensive validation suite of language conformity tests. Note that
language conformity does not ensure that, a compiler is robust, acceptably
efficient at compile time, or capable of generating correct or efficient code
for real applications. Concentrated focus on language conformity has slowed
compiler maturity along these other dimensions. The AJPO has also performed a
communication function with its Ada Information Clearinghouse.

\subsubsection*{Definition of the Ada language and development of compilers has
been successful; the next step is to implement DoD applications in Ada.}

This step is mainly acquisition management and is discussed elsewhere. AJPO can
best assist by providing truthful, complete, and candid information about
compiler and application activity.

\subsubsection*{The next technical step is to develop Ada support tools beyond
compilers and to integrate them with one another and with the underlying
operating system.}

The unclear boundary between the AJPO and the STARS programs’ charters has led
to some confusion of who should develop what support software technology.

\subsubsection*{\textit{Ada support tools}}

\subsubsection*{Ada has been overpromised.}

The Ada language embodies much current software technology. But to build
application code that is portable and reusable requires disciplined use of the
new engineering practices and tools as well. Such support tools are not yet
integrated with the compilers. Support tools are needed for such activities
as:

\begin{itemize}
    \item software documentation writing and formating;
    \item version and configuration control of both software and documentation;
    \item maintaining development history in a way that links requirements, design specification, code documentation, source code, compiled code, problem reports, code changes, tests, and test run results;
    \item debugging; and
    \item project schedule and effort management.
\end{itemize}

\subsubsection*{As a consequence, non-technical managers of programs are
expecting results that no high-level language can by Itself deliver.}

Environments that integrate such tools are not yet available for Ada. They are
likewise available only piecemeal or not at all for Jovial, C, CMS, Fortran,
etc.

\subsubsection*{Acquiring these environments is the next step.}

The DoD is supporting efforts to develop such environments. Environment design
is more difficult than language definition. Efficient environments depend
integrally upon the host operating system, yet one wants tool portability and
language independence. We expect that market forces will produce a variety of
environments around Ada if the DoD maintains firm commitment to the language.

\subsection*{Recommendations}

\subsubsection*{\textit{Recommendation 5: Commit DoD management to a serious
and determined push to Ada.}}

Management waffling is more likely to cause a failure in Ada than are technical
or acceptance problems.

Specifically, the DoD should

\begin{itemize}
    \item Finalize and issue software language policy which reaffirms and
        details the policy set forth in the DeLauer and Hicks memoranda, and
        the Weinberger speech.
\begin{itemize}
    \item The DoD should establish Ada as the common, machine-independent,
        mission critical computer system programming language for DoD-wide use.
        The mandate should not be limited to embedded computers in weapon
        systems.
    \item Each DoD component should develop and implement a plan for cutting
        over to full Ada usage. These plans should provide for support
        software, education, and training of military, technical, and
        management personnel.
\end{itemize}
    \item Stiffen practices for granting exceptions from Ada policy so that
        exceptions are difficult now and become increasingly difficult with
        time.
    \item Mandate that where implementation exceptions are granted, software
        should nevertheless be designed using Ada as the design language.
\end{itemize}

\subsubsection*{\textit{Recommendation 6: Move the Ada Joint Program Office
Into the same organizationas STARS and the SEI.}}

The major objective for Ada has become one of implementation — using the Ada
language for DoD systems — now that the AJPO has technical control of the
language. Common management of these three programs will strengthen each and
permit easy coordination of common goals and objectives.

\subsubsection*{\textit{Recommendation 7.: Keep the AJPO as the technical staff support agent
for the DoD’s executive agent.}}

Specifically, it should:

\begin{itemize}
    \item Continue firm control of the Ada language definition, permitting only
        minimal, if any, change to the now-stable language for the next two
        years.
    \item Continue Ada compiler validation. In enlarging the validation suite,
        give priority to adding tests that are representative of real
        applications.
    \item Encourage the definition of Ada working vocabularies. Promulgate
        them.
    \item Change the AJPO communication function to reduce the overselling of
        Ada and to gather and report accurate, credible information on Ada
        tools and usage. It should:
\begin{itemize}
        \item provide information on the existence and performance of Ada
            compilers and environments.
        \item document experiences of the application of Ada including both
            success stories and lessons learned.
        \item disseminate significant Ada information via newsletters, on-line
            data bases, books, articles, workshops, conference presentations,
            and videotapes.
        \item continue to act as a clearing house recording availability of
            existing and reusable Ada packages, or even entire tools, and
            objectively reporting on the experience with that software.
\end{itemize}
    \item Continue the effort to establish a performance test suite as a
        companion to the language conformity test suite.
    \item Locate software measurement techniques and tools. Publicize them and
        make these tools and techniques available to project managers using
        Ada.
    \item Initiate significant measuring and recording of lifecycle costs for
        several major Ada application programs.
    \item Continue to encourage the development of programmer support
        environments built on Ada, but be slow to standaidize environments. Let
        a winner emerge first. In particular, ALS, ALS/N, AIE and CAIS should
        not be standardized until and unless experience with prototypes shows
        implementations to be effective and efficient.
\end{itemize}

\subsubsection*{\textit{Recommendation 8: DoD policy should continue to forbid subsetting of the Ada
language.}}

There must be only one definition of the language. Further, all compilers
should correctly process the entire language; the validation process should
continue. Without this, much of the benefits of standardization will be lost.
However, organizations and educators should be encouraged to establish and
publish useful working vocabularies to simplify the task of learning and using
Ada.

Specific working vocabularies may change as the technology matures. We see
three categories - acceptable vocabularies, vocabularies to be used with
certain constraints, and vocabularies that might not be used until the
compilers and runtime environments mature. Tasking exemplifies the second
category. The use of tasking has to depend upon the performance capability of
the specific compilers. In the third category, today the Use statement might be
limited because an inordinate amount of recompile time is needed.

\subsubsection*{\textit{Recommendation 9: The DoD should increase investment in
Ada practices education and training, for both technical and management
people.}}

Each DoD component’s implementation plan should include provisions for
extensive, in-depth Ada education and training. Do not underestimate the
education and training required for managers, analysts, and administrators, in
addition to that for software engineers.

\subsubsection*{\textit{Recommendation 10: Allow fourth-generation languages to
be used where the full life-cycle cost-effectiveness of using the language
measures more than tenfold over using a general-purposelanguage.}}

Marginal increases should not dictate using such languages, especially for
long-lived, production software. The estimated cost of a program element built
in a fourth generation language must include full life-cycle cost, including
development, integration of the program element with others built in Ada, as
well as the increased maintenance costs of support software written in multiple
languages.

\section{Strategic Defense Initiative Software}

\subsection*{Findings}

\subsubsection*{The Strategic Defense Initiative (SDI) has a monumental
software problem that must be solved to attain the goals of the initiative.}

It is critical quite out of proportion to its cost, because hardware has high
replication costs and software does not. Initial contractor proposals therefore
largely ignored it.

\subsubsection*{The software problem has already received considerable public
attention and notoriety.}

\subsubsection*{No program to address the software problem is evident.}

\subsection*{Recommendations}

\subsubsection*{\textit{Recommendation 11: Focus a critical mass of software research effort on
the software needs that are unique to the SDI objectives.}}

The SDI should use what STARS, the SET, DARPA and industry produces. Much of
the software problem faced by the SDI is due to the magnitude of the required
software and the complexity of controlling the interactions of so many
components with very rapid communication and response. This is not a unique
requirement. The SDI should determine what portion of their software problem is
unique and concentrate its attention on solving the SDI-unique problem, not the
general software technology problems.

\subsubsection*{\textit{Recommendation 12: Use evolutionary acquisition, including simulation
and prototyping, as discussed elsewhere in this report, to reduce risk.}}

\section{DoD and the Civilian Software Market}
\label{sec:civ-market}

\subsection*{Findings}

\subsubsection*{The civilian market for software is today substantially larger
than the size of the DoD market, although the DoD continues to be the largest
single customer for computer software [Jorstad, 1984; Boehm, 1986].}

\subsubsection*{This new phenomenon requires a radical update in DoD thinking,
policies, and procedures.}

We find that in policy drafting and debate, the mass civilian market is
generally ignored.

One important implication is that DoD cannot, as it did with Cobol, create a
\textit{de facto} standard and impose it on the civilian market. This is not to say that
the civilian market will not adopt tools and methods from DoD where they are
perceived as advances. We merely observe that it will not adopt them Just
because DoD has.

A second implication is that DoD, although it will necessarily lead in some
aspects of the technology such as interfaces to sensors and effectors, cannot
expect to lead in most aspects of software technology development.

A corollary is that wherever DoD’s software methodology diverges from what the
civilian market is evolving by competitive selection, it will have to support
its own idiosyncratic methodology all by itself, without the resource
commitment of the larger economy.

\paragraph{Computer Security and Commercial-Off-The-Shelf-Software.} Computer
security requirements are frequently cited as a reason why commercial
off-the-shelf software cainot be used. The National Computer Security Center in
NSA has published criteria for assessing the effectiveness of security controls
in ADP systems (DoD CSC-STD-001-83).  The Center is also working on guidelines
for matching the appropriate level of security controls to a problem. With
support from the National Computer Security Center, industry is now developing
operating systems, trusted computer programs, guards, and other software and
hardware products with security controls built in to the hardware and software
and is seeking certification for these products. This approach will provide
more standard and cheaper ways of dealing with computer security than the
current practices of custom-tailoring systems. It will also facilitate
integration of computer security controls, with communications security systems
through the use of low-cost encryption devices and standard interfaces for
network control.

\paragraph{Enlisting Industry and Universities.} Our Terms of Reference
explicitly charged us with suggesting how DoD can enlist the efforts of
universities and industry in its software technology advance. Our response must
be that this charge itself assumes the bygone situation. It is now the case
that DoD, in mapping its software thrust, must in part assess which way the
technical thrust of the larger community is going, and diverge only when
absolutely necessary.

On those aspects where DoD is truly innovating technically, it can enlist the
larger community by the very excellence of the innovations — the competitive
marketplace is responsive to innovation.

\subsection*{Recommendations}

\subsubsection*{\textit{Recommendation 13: The Undersecretary of Defense
(Acquisition) should adopt a four-category classification as the basis for
acquisition policy}}

We see vast differences in the software systems that DoD buys and builds. We
recommend that these differences should be explicitly recognized by an official
classification into four major classes according to uniqueness and novelty.
Acquisition guidance, policies, and procedures should be framed separately for
each class. The classes are:

\begin{center}
    \begin{tabular}{ l l }
Standard: & Off-the-shelf, commercially available \\

Extended: & Extensions of current systems, both DoD and commercial \\

Embedded: & Functionally unique and embedded in larger systems \\

Advanced: & Advanced and exploratory systems. \\
    \end{tabular}
\end{center}

Each Program Manager would classify his system, its subsystems, major
components, major increments, and phases into one of these classes, with the
burden of proof being to show why the element should be in a higher class
instead of a lower one.

% The landscape environment creates a new page, so this needs to be placed
% approximately where LaTeX would otherwise insert a page break.

\begin{landscape}

% Table 5.1 -- category attributes

\begin{center}
\begin{multicols*}{4}
\raggedcolumns % Avoid trying to equalize the columns, we don't care
\raggedright

    \textbf{STANDARD}
    \begin{itemize}
        \item Commercial Hardware
        \item Commercially Available Operating System
        \item Commercially Available Application System
        \item Fully Reusable
        \begin{itemize}
            \item Specification
            \item Documentation
            \item Code
        \end{itemize}
    \end{itemize}

    \columnbreak

    \textbf{EXTENDED}
    \begin{itemize}
        \item Commercial Hardware
        \item Commercially Available Operating System
        \item Commercially Available Software and Custom Extensions
        \item Reusable:
        \begin{itemize}
            \item Specifications
            \item Algorithms
            \item Source Code
            \item Object Code
        \end{itemize}
        \item Some use of a Prototyping Facility
    \end{itemize}

    \columnbreak

    \textbf{EMBEDDED}
    \begin{itemize}
        \item Commercial and Special-Purpose Hardware
        \item Commercial and Special-Purpose Operating System
        \item Special-Purpose Application Software
        \item Reusable:
        \begin{itemize}
            \item Specifications
            \item Designs
            \item Algorithms
            \item Code
        \end{itemize}
        \item Use of a Prototyping Facility
        \item Risk Management Discipline
        \item Heavy User Involvement in Evolutionary Acquisition
    \end{itemize}

    \columnbreak

    \textbf{ADVANCED}

    \begin{itemize}
        \item State-of-the-Art and Uniquely Designed Hardware
        \item[] ~
        \item[] ~
        \item Uniquely Designed Software
        \item Some Reuse:
        \begin{itemize}
            \item Specifications
            \item Algorithms
        \item[] ~
        \item[] ~
        \end{itemize}
        \item Use of a Prototyping Facility
        \item Risk Management Disciplines are Critical
        \item Heavy User Involvement in Evolutionary Acquisition
    \end{itemize}

\end{multicols*}
\end{center}

\newpage

% Table 5.2 -- category examples in DoD acquisition

\begin{center}
\begin{multicols*}{4}
\setstretch{1.4} % Space out the entries a bit
\raggedcolumns % Avoid trying to equalize the columns, we don't care
\raggedright

    \textbf{STANDARD}

    Defense Data Network Initial Phases

    WWMCCS Information System Block A

    \columnbreak

    \textbf{EXTENDED}

    Defense Data Network Initial Phases

    WWMCCS Information System Block A

    Airborne Warning and Control System (AWACS)

    Army WIS (AWIS)

    AF WIS (AFWIS)

    Army Operations Center Upgrade

    Joint Automatic Message Processing System

    AFLC Local Area Network

    Intelligence Data Handling System (IDHS)

    Pentagon Local Area Network

    Community Intelligence Network System (COINS)

    DoD Intelligence Information Systems (DODIIS) Local Area Network

    \columnbreak

    \textbf{EMBEDDED}

    MILSTAR

    JTIDS

    AWACS

    High Frequency Anti-Jam Program

    ARC-182-ECCM Radio

    Global Position System

    Communications System Control Environment (CSCE)

    Integrated Automated Intelligence Processing System

    Joint Tactical Fusion

    Command Center Processing and Display System

    SACDIN

    SEEK IGLOO Radar

    PAVE PAWS Radar

    Advanced Combat Direction System-Navy

    AEGIS Defense System

    \columnbreak

    \textbf{ADVANCED}

    Strategic Defense Initiative

    Strategic Computing Intiative Program

\end{multicols*}
\end{center}

\end{landscape}

Table 5.1 provides attributes of the four categories. Table 5.2 illustrates the
categories by classifying some current acquisitions.

It may also be wise to establish categories by size (lines of source code) for
uniformity in description. A possibility might be:

\begin{center}
    \begin{tabular}{ l l }
Modest: & Under 2000 LOSC\\

Small: & 2000-10,000 LOSC\\

Medium: & 10-100 KLOSC\\

Large: & Over 100 KLOSC\\
    \end{tabular}
\end{center}

Then a planned undertaking could be characterized for policy or procedure purposes
as, for example, a “Medium Extended-Class software project.”

% NOTE: Page 26, 27 of the original PDF contain landscape formatted tables
% which appear to be Table 5.1 and 5.2 as mentioned in the text above. I don't
% see where they appear in the auto-scraped PDF text quite yet but this is
% where they would go, in theory.

\subsubsection*{\textit{Recommendation 14: The Undersecretary of Defense
(Acquisition) should develop acquisition policy, procedures, and guidance for
each category.}}

\subsubsection*{\textit{Recommendation 15: The Undersecretary of Defense
(Acquisition) and the Assistant Secretary of Defense (Comptroller) should
direct Program Managers to assume that system software requirements can be met
with off-the-shelf subsystems and components until it is proved that they are
unique.}}

The \emph{cheapest} way to get software is to buy it in the commercial
marketplace rather than to build it.

The \emph{fastest} way to get software is to buy it in the commercial
marketplace rather than to build it.

The \emph{surest} way to get robust, maintained, supported software is to buy
it in the commercial marketplace rather than to build it.

Even though commercial software is often delinquent in these latter respects,
competitive pressures get the delinquencies fixed. Custom-built software has
historically been notoriously bad in these respects, without the same pressures
to fix it.

Hence the DoD-wide assumption should be that a commercial product will be
usable if the function is similar to that required — perhaps with modifications
or extensions, perhaps with extra documentation, perhaps with different support
and maintenance arrangements.  The first investigation when requirements begin
to be formulated should be into the market, to see what is available already.
Even if the best off-the-shelf product is not ultimately used, adopting it for
pilot use will help radically in setting specifications for the custom product.

\subsubsection*{\textit{Recommendation 16: All the methodological efforts,
especially STARS, should look to see how commercially available software tools
can be selected and standardized for DoD needs.}}

Although end-user systems, especially embedded ones, will often need to be
specialized, and perhaps custom-built, it will rarely be justified for DoD to
custom-build the tools with which its software is built. The assumption should
be that marketplace tools will be used.

\section{DoD In a Sellers’ Market for Software}

\subsection*{Findings}

\subsubsection*{Because of the explosion of the commercial software market, DoD
now Is In a sellers’ market for software-building.}

Just as the DoD need for software is growing exponentially, and its
software-skilled personnel grow more slowly, the same is true of the commercial
software market. Programmers are in short supply, programming managers even
scarcer, and software system designers very scarce. Hence the companies that
have these skills, and have them organized into functioning, equipped teams,
have many choices as to how best to market their services.

\subsubsection*{DoD is perceived as a poor customer, and the stable of DiD custom software
vendors stays small even though the requirement grows radically.}

\paragraph{Poor Return.} Building custom software for DoD has a poor profit
margin. In calculating proper profit levels for cost-plus-incentive contracts,
DoD tends to use the same margin for software development as for hardware
development, although the latter is customarily followed by a production cycle
at acceptable total profit levels. Ten percent profit on sales is considered
high in DoD, whereas it is grossly unacceptable in computer industry pricing on
software.

\paragraph{Weak Incentives for Productivity.}  Even on fixed-price software
contracts, there are only weak incentives to manage for higher productivity or
to invest company money in capital tooling that will save labor cost. High
productivity and high quality are not rewarded by DoD except at the time of
contractor selection. If “excessive” profits result from high productivity on
firm-fixed-price contracts, the profits are readjusted after audit. The
standard for “excess” is the same as that for hardware development, despite the
absence of the production cycle.

\paragraph{Heavy Regulation.} DoD Directive 5000.29, “Management of Computer
Resources in Major Defense Systems”, DoD STD 2167, “Defense System Software”,
and the proposed new Federal Acquisition Regulation FAR 27.4, “Data Rights”,
and the proposed DoD FAR Supplement 27.4 have as their purpose to ensure fair,
consistent, and open competition, and to get the most capability for each
dollar spent. In practice, however, they inhibit the use of standard commercial
practices in software acquisition and maintenance. They also encourage the
building of new software rather than purchase of off-the-shelf items.

\paragraph{Unreasonable Rights-In-Data Requirements.}  DoD FAR Supplement 27.4,
as proposed for public comment by 10 January 1986, set forth demands for DoD
ownership of rights to programs, documentation, tools, and methods that were

\begin{itemize}
    \item formulated only in terms of hardware, entirely ignoring the different circumstances of software,
    \item completely at variance with commercial practice for software,
    \item probably illegal under U.S. Copyright law, and
    \item destructive of most vendor incentive to hivest in better tools and methods.
\end{itemize}

We were convinced that the misfit between the proposed supplement and software
is entirely unintentional, and so the Task Force made timely presentations to
the Undersecretary of Defense (Acquisition) and to the Assistant Secretary for
Acquisition and Logistics during the comment period. If the proposed supplement
were to be adopted, however, it would be another powerful disincentive for
vendors to bid on DoD custom software development.

\paragraph{Constant, Small, Stable of Vendors.} Given the regulatory and
financial structure of DoD contracting outlined above, a vendor can participate
in substantial DoD software contracting only if it:

\begin{itemize}
    \item has staying power to average over crests and troughs in contracting
        business,
    \item has a management superstructure to cope with the regulatory overhead,
        including a Program Control Management System, and specialized
        accounting to meet DoD STD 7000.2,
    \item has an infrastructure of technical specialists to deal with
        configuration management per regulation, documentation per regulation,
        acceptance testing, etc.
    \item therefore has a critical mass of skilled people so that it can carry
        several contracts at once, both to smooth crests and troughs and to pay
        for the administrative and technical superstructure necessary to cope
        with the regulatory overhead.
\end{itemize}

As a result, we estimate there to be only about two dozen houses that are
regularly available to participate in the development of substantial
mission-critical software systems, and this number grows slowly. On the
contrary, several vendors are seriously considering leaving the DoD business
entirely, and refusing to bid in the future.

The net result of this small stable of vendors in a time of exponential growth
of work is sure to be higher bids and longer schedules. It is in DoD’s
interests, and the nation’s interests, for DoD to make itself an attractive
customer. We believe the net costs to the nation of weapon-system software will
be lower if it does. Present practices are penny-wise and pound-foolish in many
petty ways.

\subsection*{Recommendations}

\subsubsection*{\textit{Recommendation 17: DoD should devise increased productivity incentives
for custom-built software contracts, and make such incentivized contracts the
standard practice.}}

A new contracting form, part-way between fixed-price and cost-plus-fee, should
be devised. For instance, on a cost-type contract, a productivity figure is
usually bid.  Competition in vendor selection keeps the figure from being
unreasonably low. So reimbursement might be structured to split any savings due
to increased productivity evenly between the buyer and the seller.

Another new contracting form that we recommend DoD consider would be to
guarantee a quantity buy of some software product and to request bids solely on
a per-copy price. Here the vendor would bear all the non-recurring costs and
recover gains on capital investment and productivity enhancement. Ada compilers
and software development and maintenance environments are examples that could
be purchased or licensed this way.

\subsubsection*{\textit{Recommendation 18: DoD should devise increased profit incentives on
software quality.}}

One such incentive could be a sliding profit margin based on the quality of the
delivered complete software product. This requires quality metrics, recommended
below.

Another kind of incentive could be introduced by requesting contractors to bid
a per-copy-per-year fixed license fee including maintenance. In this way, high
quality resulting in low maintenance would provide financial rewards to the
contractor and operational rewards to the users.

\subsubsection*{\textit{Recommendation 19: DoD should develop metrics and
measuring techniques for software quality and completeness, and Incorporate
these routinely in contracts.}}

There are today no metrics for source-code quality, object-code quality,
documentation quality, etc. Part of the STARS methodological effort should be
addressed to such metrics, for Ada programs in particular.

Meanwhile, there are techniques for judging the over-all quality of complex
performances outside the computer field. There is wide agreement as to what
quality is, and skilled practitioners make similar judgements when presented
with products to judge. Even today software quality can be judged by panels of
trained judges, just as such panels judge Olympic diving, skating, and
acrobatic performances. DoD should immediately begin testing such panel methods
for consistency and reliability, and, if they work, begin using such judgements
for quality incentives.

DoD buys software for operational systems that will be used for a decade or
more. The software must be maintainable and changeable. DoD is willing to pay
the price for complete software products, but all too often it accepts less
because of schedule slippages and operational needs.

Complete software products should be mandated in contracts to include:

\begin{itemize}
    \item specifications the describe the actual software structure as built,
        as opposed to that originally specified,
    \item documentation showing the structure and organization of the software,
    \item source code that is properly structured, well modularized, and well
        commented, especially in procedure headers and variable declarations,
    \item cross-reference documentation that traces articles in the
        specifications to the corresponding source code, and vice-versa.
\end{itemize}

\subsubsection*{\textit{Recommendation 20: DoD should develop metrics to
measure Implementation progress.}}

Such metrics would help ensure that costs and schedules are being met and that
complete products will be delivered. They might include, for example, program
size, software complexity metrics, personnel experience, testing progress, and
incremental-release content. Development of such has in the past been part of
the STARS plan; it should continue to be. Meanwhile, panel-judging techniques
as discussed above can be applied to progress as well as to quality.

\subsubsection*{\textit{Recommendation 21: DoD should examine and revise
regulations to approach modern commercial practice insofar as practicable and
appropriate.}}

\subsubsection*{\textit{Recommendation 22: DoD should follow the concepts of
the proposed FAR 27.4 for data rights for military software, rather than those
of the proposed DoD Supplement 27.4, or it should adopt a new “Rights In
Software” Clause as recommended by Samuelson, Deasy, and Martin in Appendix
A6.}}

The legal problem is highly technical. Two good solutions, the arguments for
proposed FAR 27.4, the concerns about the clarity and applicability of the
proposed DoD Supplement 27.4, and the arguments for a new clause are all
skillfully set forth in the SEI Technical Report incorporated herein as
Appendix 6A.

To those technical arguments we would add an economic one: the proposed DoD
Supplement 27.4 is, intentionally or otherwise, grabby in spirit and effect. No
fair-minded person would consider it to propose equitable treatment among
vendors, or between DoD and vendors. Its lack of clarity and clumsiness of
drafting will occasion litigation. The net effect will be to further shrink the
vendor pool, at great cost to DoD and the taxpayer.

\section{A New Life-Cycle Model for Custom DoD Software}

\subsection*{Findings}

\subsubsection*{The most common present method of formulating specifications —
issuing a Request for Proposal, accepting bids, and then letting a contract for
software delivery — is not in keeping with good modern practice and accounts
for much of the mismatch between user needs and delivered function, cost, and
schedule.}

As discussed above under Current Trends, we now understand the importance of
iterative development of requirements, the testing of requirements against real
users’ needs by rapid prototyping, and the construction of systems by
incremental development, with early incremental releases subjected to
operational use.

\subsubsection*{The Task Force finds that Directive 5000.29 and STD 2167, as
interpreted, have made it difficult to apply these modern methods.}

Although some parts ef the recent Draft DOD-STD-2167A appear to encourage
modern methods, the draft as a whole continues to reinforce exactly the
document-driven, specify-then-build approach that we believe causes so many of
DoD’s software problems.

\subsection*{Recommendations}

\subsubsection*{\textit{Recommendation 23: The Undersecretary of Defense
(Acquisition) should update DoD Directive 5000.29, “Management of Computer
Resources In Major Defense Systems”, so that it mandates the iterative setting
of specifications, the rapid prototyping of specified systpems, and incremental
development.}}

We propose that the iterative development, of specifications can be reconciled
with the needs of fair and open competition by letting two level-of-efforts
contracts for the specifying and prototyping of major software systeLms. After
prototyping is complete and specifications formulated, a software production
contract can be put out for bidding in the usual process.

An alternative for the specification process is to let a separate contract to a government
contractor for specification, with the specifying contractor excluded from bidding on the
build.

The iterative development of specifications is a small part of the total cost
of a major software system, usually less than 10\%, but it has profound effects
on the procurement cost, life-cycle cost, schedule, and function of the
product.

We believe the procurement cycle should be modified so that the requirements
remain unfrozen and subject to alteration until the cost and performance
effects of their provisions can be known from early product design. This means
final requirements would not be frozen until perhaps one-third of the way
through the procurement period, a substantial departure from present practice.

\subsubsection*{\textit{Recommendation 24: DoD STD 2167 should be further revised to remove
any remaining dependence upon the assumptions of the “waterfall” model and
to institutionalise rapid prototyping and incremental development.}}

\subsubsection*{\textit{Recommendation 25: Directive 5000.29 and STD 2168 should be revised
or superseded by policy to mandate risk management techniques In software
acquisition, as recommended In the 1983 USAF/SAB Study.}}

The Air Force Scientific Advisory Board in 1983 identified software risk
factors and recommended risk management techniques [Munson, 1983]. While parts
of the recommendations are Air-Force-specific, the ideas are applicable to all
Services. An example of how these risk management techniques have been
incorporated into program management is included in table
\ref{table:risk-mgmt-plan} below. The risk-management approach provides an
effective way for a project to determine when, where, and how much to use
prototyping and similar risk-reduction techniques.

% Fake that we're a previous section so table numbers right
\addtocounter{section}{-1}

\begin{table}[hb]
    \caption{Software Risk Management Plan}
    \label{table:risk-mgmt-plan}
\begin{enumerate}
    \item Identify the project’s top 10 risk items.
    \item Present a plan for resolving each risk item.
    \item Update list of top risk items, plan, and results monthly.
    \item Highlight risk-item status in monthly project reviews.
    \item Initiate appropriate corrective actions.
\end{enumerate}
\end{table}

% Undo the fake section numbering again
\stepcounter{section}

\subsubsection*{\textit{Recommendation 26: Each Service should provide its
software Product Development Division with the ability to do rapid prototyping
In conjunction with users.}}

The DoD software system acquisition agents are the service product development
divisions. Each of these divisions needs facilities and equipments to mock-up,
simulate, and build critical prototypes of the new systems being acquired. Some
of the system interfaces can be tested prior to delivery to the users. Such
facilities can also serve as a place for the developer and user to meet and
refine requirements and procedures. The placement of such facilities at the
product development division level will allow their use by multiple Program
Managers who all report to the same local commander. The product development
divisions are organized along mission categories, and programs in each division
will tend to need equipment and software with similar power and capabilities.
The facilities could also be used for Ada training.

\subsubsection*{\textit{Recommendation 27: Each Service should provide its
software Using Commands with facilities to do comprehensive operational testing
and life-cycle evaluation of extensions and changes.}}

The user commands are responsible for defining the original requirements.
However, many of the systems that are being developed are doing jobs that have
never been done before. These types of systems tend to be technology-driven and
must be placed in the hands of the user as early as possible to develop new
operational procedures. In the early phases of system acquisition, a facility
at the user command is needed for testing, evaluation, training, and procedure
development. Throughout the life of the system, the user commands need the
facility for testing and evaluation of changes and upgrades to systems.

\subsubsection*{\textit{Recommendation 28: The Undersecretary of Defense
(Acquisition) and the Assistant Secretary of Defense (Comptroller) should by
directive spell out the role of Using Commands in the evolutionary and
incremental development of software systems.}}

The relationships between Developing Commands and Using Commands for the
different kinds of systems should be spelled out in policy statements. The role
and responsibilities of the user commands can vary with the system acquisition
procedures and the kind of system being acquired. In conventional acquisitions
such as weapons, platforms, or sensor systems, a system is developed, tested,
and turned over to the user.  For command and intelligence systems, much of the
development and testing can take place at the user command. The experience of
the users with the first capability built can be used (or required) as feedback
to the second increment of the system. User involvement is obviously heavier
when evolutionacy acquisition procedures are used.

\section{Module Reuse in DoD Custom Software}

\subsection*{Findings}

\paragraph{Software technology now enables the extensive reuse, even in mission-critical
embedded systems, of software modules written for other systems.}

% Sacrificial paragraph to join with previous para.
\,\par

The typical module size is on the order of one to two pages of source code,
25-50 source lines. Ada in particular allows modules to be used easily and
safely, because the module interface is packaged and specified separately from
the body, which the user does not ordinarily need to inspect or alter.

\paragraph{Module reuse requires new forms of contractor incentives, both to make modules
available for others to use, and to use them themselves instead of building
anew.} Making a module reusable requires a modest extra effort in design and
development. There has to be incentive and compensation for this effort.

\paragraph{Module reuse requires the establishment of clearinghouses or markets where
modules can be exchanged.}

\paragraph{Module exchange requires the establishment of standards of description of
function and of degree of testing.}

\subsection*{Recommendations}

\subsubsection*{\textit{Recommendation 29: The Undersecretary of Defense
(Acquisition) should develop economic incentives, to be incorporated into
standard contracts, to allow contractors to profit from offering modules for
reuse, even though built with DoD funds.}}

\subsubsection*{\textit{Recommendation 30: The Undersecretary of Defense
(Acquisition) should develop economic incentives, to be incorporated Into all
cost-plus standard contracts, to encourage contractors to buy modules and use
them rather than building new ones.}}

Acquisition contracts should be structured so that contractors will be
motivated to build and sell reusable software, and to buy it. Reusable software
will be successful when contractors decide it is in their competitive
self-interest to reuse software rather than to develop it each time. The proper
incentives with respect to data rights, warranties, licenses, liabilities, and
maintenance must be included in the RFPs and the contracts.

\subsubsection*{\textit{Recommendation 31: The Undersecretary of Defense
(Acquisition) and Assistant Secretary of Defense (Comptroller) should direct
Program Managers to identify in their programs those subsystems, components,
and perhaps even modules, that may be expected to be acquired rather than
built; and to reward such acquisition In the RFP’s.}}

\subsubsection*{\textit{Recommendation 32: The Software Engineering Institute
should establish a prototype module market, focussed originally on Ada modules
and tools for Ada, with the objective of spinning it off when commercially
viable.}}

A scheme for how such a marketplace might work, including some possible
financial and licensing arrangements, is proposed in Appendix A7.

The White Sands Missile Range is operating today an Ada Software Repository,
apparently using volunteer labor and spare computer capacity. We believe that a
more regular and reliable service must be based upon licensing and license
fees, and our proposal includes that. Rudimentary validation of compilability
by the marketer may also be necessary.

\subsubsection*{\textit{Recommendation 33: The Software Engineering Institute,
in consultation with the Ada Joint Program Office, should establish standards
of description lor Ada modules to be offered through the Software Module
Market.}}

\section{Software-Skilled People}

DoD’s demand for software capability grows exponentially. It does so at a
greater rate than the combined growth in the size and productivity of the pool
of software personnel.

Previous DoD studies have identified personnel issues as critical elements of
DoD’s software problems. These studies have made excellent recommendations on
personnel issues, but the recommendations have not been acted upon.

The Task Force recommends a new approach to the software personnel problem.

\subsection*{Findings}

\subsubsection*{Previous Studies Have Made Good Recommendations}

A number of previous studies of the DoD software problem have identified the
scarcity of in-house DoD software personnel as a critical problem. They have
developed similar sets of recommended actions for dealing with the problem,
e.g.:

\begin{itemize}
    \item Determine DoD’s needs for the various software-related skills,
    \item Create and maintain a skills inventory for DoD personnel,
    \item Create and implement more attractive career paths for DoD software
        personnel,
    \item Establish educational programs to support these career paths.
    \item Analyze the factors influencing the development and retention of DoD
        personnel with the appropriate mix of software-related skills,
\end{itemize}

\subsubsection*{Previous Recommendations Have Not Been Acted Upon}

If these actions were vigorously pursued, they would go a long way toward
solving the problem.

However, for various reasons, DoD and the Services have not acted on these
previous recommendations. It is therefore unlikely that any effective action
would result from yet another restatement of these recommendations by this Task
Force.

\subsubsection*{We believe the pool of DoD software personnel has remained about the same size
for many years.}

\subsubsection*{The national pool of software personnel Is growing rapidly.}

The number of Bachelor’s and Master’s Degrees in Computer Science, Mathematics,
and Statistics are shown in Table \ref{table:cs-degrees}. Historically, the
supply of computer science graduates has been augmented primarily from
graduates in Mathematics. These sources of new personnel meet requirements
estimated at 54,000 new graduates per year in 1983 [Hamblen, 1984] to sustain a
pool of computer specialists whoee size was estimated at 299,000 in 1982
[Vetter, 1985] and whose annual growth rate is estimated at about 5\% [NSF,
1984].

\subsubsection*{It appears that DoD is not competing effectively with the
private sector in attracting and retaining software personnel.}

% Fake that we're a previous section so table numbers right
\addtocounter{section}{-1}

\begin{table}[ht]
    \caption{Degrees In Math/Statistics and Computer Science, 1970-82}
    \label{table:cs-degrees}
\begin{center}
    \begin{tabular}{ c c c c c }
        \multicolumn{3}{c}{Bachelor’s Degrees} & \multicolumn{2}{c}{Master’s Degrees}\\
        Year & Math/Statistics & Computer Science & Math/Statistics & Computer Science\\
        1970 & 27,442 & 1,544 & 5,636 & 1,459\\
        1971 & 24,801 & 1,624 & 5,121 & 1,588\\
        1972 & 23,713 & 3,402 & 5,198 & 1,977\\
        1973 & 23,067 & 4,305 & 5,028 & 2,113\\
        1974 & 21,635 & 4,757 & 4,834 & 2,276\\
        1975 & 18,181 & 5,039 & 4,327 & 2,299\\
        1976 & 15,984 & 5,664 & 3,857 & 2,603\\
        1977 & 14,196 & 6,407 & 3,695 & 2,798\\
        1978 & 12,569 & 7,224 & 3,373 & 3,038\\
        1979 & 12,115 & 8,769 & 3,578 & 3,055\\
        1980 & 11,473 & 11,213 & 2,868 & 3,647\\
        1981 & 11,078 & 15,121 & 2,567 & 4,218\\
        1982 & 11,599 & 20,267 & 2,727 & 4,935\\
    \end{tabular}

    Source: Table 12 of [Vetter, 1985]
\end{center}
\end{table}

% Undo the fake section numbering again
\stepcounter{section}

\subsubsection*{DoD needs software talent primarily to support the acquisition
process}

We agree with previous studies that the software personnel shortage hurts DoD
most in the area of software acquisition management.

MITRE and TRW experience have found software acquisition has been most
effective when DoD had an acquisition management cadre whose size is roughly
10\% (5-15\%) of the size of the developer’s staff; a cadre with a thorough
understanding of software technology and acquisition management.

The people in such a cadre are not just watchers. They add considerable value
to the software product by developing specifications, operational concept
documents, and life cycle plans; managing competitive concept definition,
preparing precise RFP packages, performing thorough source selections,
including pre-award audits and independent cost estimates; exercising
prototypes for realistic user feedback; improving specifications, plans and
manuals; monitoring the effective performance of quality assurance,
configuration management; and subcontracting and financial management,
representing user interests on change control boards.

\subsubsection*{DoD does not have adequate career paths for software professionals}

Some Services have no career paths; some Services alternate computer
assignments with totally unrelated non-computer assignments, thereby diffusing
the officer’s experience. In many cases, software expertise is encoded as a
subspecialty inflection rather than in a primary specialty code.

Software engineering methods and techniques are advancing dramatically. It is
critical for software professionals to master these advanced methods and
techniques and to keep learning new techniques as they are developed. This is
as important for acquisition people as it is for production people. (Building
architects have to know the technologies better than most contractor people.)
In-house software skills do not match those of top contractor pools.

\subsubsection*{Current deployment of the software talent pool is ineffective}

Currently, many software-qualified personnel are assigned to jobs that could
effectively be assigned to contractors. Many DoD software acquisitions are
either in-house development efforts staffed entirely with DoD personnel or
contracted acquisitions with DoD staffing levels far below the needed 5-15\%.

\subsection*{Recommendations}

\subsubsection*{\textit{Recommendation 34: Do not believe DoD can solve its
skilled personnel shortage; plan how best to live with it, and how to
ameliorate it.}}

The software personnel shortage will not disappear by direct DoD action. All
DoD plans should be based on the assumption that an acute skill shortage will
persist.  Organization structures should be tuned, assignment policies should
be adjusted, and educational programs should be revised to produce the military
and civilian cadre needed to acquire and maintain highly complex systems.

Further, DoD should facilitate supplementing the software acquisition
management process with contractor support where the supply of in-house
personnel is insufficient.

\subsubsection*{\textit{Recommendation 35: Use DoD people for acquisition
instead of construction.}}

Instead of hoping that enough personnel can be hired and retained to satisfy
the needs of the current strategy for using software personnel, change that
strategy to train and assign available personnel to the highly-leveraged tasks,
namely software acquisition management. DoD should sharply reduce in-house
software construction, extension, and maintenance, limiting such to critical
functions at operational bases, adaptation of existing software to local needs,
and special security-sensitive work.

\subsubsection*{\textit{Recommendation 36: Establish mechanisms for tracking
personnel skills and projecting personnel needs.}}

No meaningful studies have been found that catalog seasoned personnel, and no
studies have been found that include uniformed personnel and government
civilians.

Each Service needs to have, and all DoD needs access to, a database that covers
its officers, its senior and technically skilled enlisted people, and its
technically skilled civilian employees. This should show not only career
history and assignments, but technical skills, experiences, and trainings by
quite fine subject codes.

From such a database each Service should not only draw for particular
assignments, but also project biennially the trends by skill, by seniority, by
median age within skill, and by years of particular skill experience. Such
trends can then be assessed against projected needs.

\subsubsection*{\textit{Recommendation 37: Structure some officer careers to
build a cadre of technical managers with deep technical mastery and broad
operational overview.}}

Where possible, operational assignments should be chosen to give intense
systemusing experience in real operations, development/acquisition assignments
should be on related software systems; and education assignments should focus
on new technical and management approaches.

\subsubsection*{\textit{Recommendation 38: Enhance education for software
personnel.}}

DoD should implement the education and training necessary for its software
acquisition management personnel to master both software technology and
acquisition management.

\section{Appendices}

\subsection*{A1. The Task Force}

\subsection*{A2. Terms of Reference}

\subsection*{A3. Meetings and Briefing}

\subsection*{A4. Documents Studied}

\subsection*{A5. Software — Why Is It Hard?}

\subsection*{A6. Proposal for a New “Rights in Software” Clause}

\subsection*{A7. Proposal for a Module Market}

\newpage

\addcontentsline{toc}{subsection}{A1. The Task Force}
\section*{Appendix A1}
\begin{centering}

\subsection*{DEFENSE SCIENCE BOARD SOFTWARE TASK FORCE}

\subsubsection*{Members and Key Persons}
\end{centering}

\textbf{Dr. Donald Hicks, USD(R\&E), Sponsor 202-695-6639}

\medskip

\textbf{Dr. James P. Wade Jr., Deputy USD(R\&E), Acting ASD(A\&L), Sponsor}

\medskip

\textbf{Dr. Victor Basili, Member}\\
\hspace*{2cm}Chairman, Department of Compute: Science\\
\hspace*{2cm}University of Maryland\\
\hspace*{2cm}College Park, Maryland 20742\\
\hspace*{2cm}301-454-2002

\medskip

\textbf{Dr. Barry Boehm, Member}\\
\hspace*{2cm}TRW Defense Systems Group\\
\hspace*{2cm}One Space Park, R2-2086\\
\hspace*{2cm}Redondo Beach, CA 90278\\
\hspace*{2cm}213-535-2184

\medskip

\textbf{Ms. Elaine Bond, Member}\\
\hspace*{2cm}Senior Vice President\\
\hspace*{2cm}Chase Manhattan\\
\hspace*{2cm}One New York Plaza, 21st Floor\\
\hspace*{2cm}New York, NY 10081\\
\hspace*{2cm}212-676-2982

\medskip

\textbf{Dr. Frederick P. Brooks, Jr., Task Force Chairman}\\
\hspace*{2cm}Kenan Professor of Computer Science\\
\hspace*{2cm}University of North Carolina at Chapel Hill\\
\hspace*{2cm}New West Hall 035A\\
\hspace*{2cm}Chapel Hill, NC 27514\\
\hspace*{2cm}919-962-2148

\medskip

\textbf{Mr. Neil Eastman, Member}\\
\hspace*{2cm}IBM Federal Systems Division\\
\hspace*{2cm}18100 Frederick Pike\\
\hspace*{2cm}Building 929, Room 1C12\\
\hspace*{2cm}Gaithersburg, MD 20879\\
\hspace*{2cm}301-240-2170

\medskip

\textbf{MGen. Don L. Evans, Member}\\
\hspace*{2cm}USAF (ret)\\
\hspace*{2cm}President\\
\hspace*{2cm}Tartan Laboratories\\
\hspace*{2cm}461 Melwood Avenue\\
\hspace*{2cm}Pittsburgh, PA 15213\\
\hspace*{2cm}412-621-2210

\medskip

\textbf{Dr. Anita K. Jones, Member}\\
\hspace*{2cm}Tartan Laboratories\\
\hspace*{2cm}461 Melwood Avenue\\
\hspace*{2cm}Pittsburgh, PA 15213\\
\hspace*{2cm}412-621-2210

\medskip

\textbf{Dr. Mary Shaw, Member}\\
\hspace*{2cm}Software Engineering Institute\\
\hspace*{2cm}Carnegie-Mellong University\\
\hspace*{2cm}Pittsburgh, PA 15213\\
\hspace*{2cm}412-578-7731

\medskip

\textbf{Mr. Charles Zraket, Member}\\
\hspace*{2cm}President\\
\hspace*{2cm}The MITRE Corporation\\
\hspace*{2cm}Burlington Road\\
\hspace*{2cm}Bedford, MA 01730\\
\hspace*{2cm}617-271-2356

\medskip

\textbf{Dr. Edward Lieblein, Government Representative, OSD}\\
\hspace*{2cm}202-694-0208

\textbf{LTC Bill Freestone, Government Representative, Army}\\
\hspace*{2cm}202-694-7298

\textbf{Mr. Marshall Potter, Government Representative, Navy}\\
\hspace*{2cm}202-697-9346

\textbf{LTC Dave Hamond, Government Representative, Air Force}\\
\hspace*{2cm}202-697-3040

\textbf{Major Susan Swift, USAF, Executive Secretary}\\
\hspace*{2cm}202-695-7181

\textbf{CDR Chris Current, DSB Secretariat Representative}\\
\hspace*{2cm}202-695-4157

\textbf{CDR Mike Kaczmarek, DSB Secretariat Representative}\\
\hspace*{2cm}202-695-4157

\textbf{Mr. Robert L. Patrick, Contractor Support}\\
\hspace*{2cm}Willow Springs Road\\
\hspace*{2cm}Star Route 1, Box 269\\
\hspace*{2cm}Rosamond, CA 93560\\
\hspace*{2cm}805-256-4444

\medskip

\textbf{Mr. John K. Summers, Contractor Support}\\
\hspace*{2cm}The MITRE Corporation W90\\
\hspace*{2cm}7525 Colshlre Drive\\
\hspace*{2cm}McLean, VA 22102\\
\hspace*{2cm}703-883-6146

\newpage

\addcontentsline{toc}{subsection}{A2. Terms of Reference}
\section*{Appendix A2 — Terms of Reference}
\begin{flushright}
    2 NOV 1984
\end{flushright}

\noindent
MEMORANDUM FOR CHAIRMAN, DEFENSE SCIENCE BOARD\\

\noindent
SUBJECT: Defense Science Board Task Force on Software\\

You are requested to form a Task Force on Software.  Software costs are
projected to increase substantially in the next decade, and the cost of
software development is becoming an increasing fraction of total development
costs of many types of weapons systems. In addition, the testing of software to
prove performance is becoming increasingly difficult and-time consuming,
leading to delays in system deployment. The need for software productivity
improvement is well recognized.

The task Force should address this broad problem and
identify and assess the following:

\renewcommand{\labelenumi}{\Alph{enumi}.}
\begin{enumerate}
    \item Assess and unify the conclusions and recommendations of the various
        recent studies of military software problems.
    \item Examine and discuss the theoretical and practical reasons that make
        software costs high including the design and analysis of tests.
    \item The probable effectiveness of the proposed DoD STARS program at
        addressing military software problems, and the relative priority of the
        components of the STARS program (suggest alternative to STARS if
        necessary)
    \item Ways of enlisting industry, Service laboratories, and university
        efforts in-programs aimed at software productivity.
    \item The probable effectiveness of the STARS program and U.S. international
        competitiveness in software production.
    \item How to use limited R\&D funds to make the biggest improvement in the
        development of military software capabilities.
    \item Implementation concepts for an incremental, evolutionary approach in
        case an all-out assault on the software problem cannot be funded.
\end{enumerate}

Based on the above assessments, the Task Force should make specific
recommendations for significant improvements in the way the DoD manages and
develops software.  I would appreciate a report in approximately six months.

Dr. James P. Wade, Jr., Assistant Secretary of Defense (D\&S), is sponsoring
this study. Dr. Frederick P. Brooks has agreed to serve as Chairman.  Major
Susan Swift, USAF, will serve as Executive Secretary.  Commander Chris Current,
USN, will be the DSB Secretariat representative.  It is not anticipated that
your inquiry will go into any “particular matters” within the meaning of
Section 208 of Title 19, U.S. Code.

\begin{center}
    /s/ R. D. DeLauer
\end{center}

\noindent Enclosure\\
\hspace*{1ex}Proposed Membership

\newpage

\addcontentsline{toc}{subsection}{A3. Meetings and Briefings}
\section*{Appendix A3 — Condensed Briefings and Minutes}


The complete minutes are on file in the DSB Office. The meetings arn abstracted
below.

~

\noindent
\begin{footnotesize}
\begin{tabular}{ l l l l l }
\textbf{Date} & \textbf{Place} & \textbf{Subject} & \textbf{Briefer} & \textbf{Organization}\\
\\
11 March ’85 & Pentagon & Conflict of Interest & David Ream & DoD Counsel\\
    & & Army Study    & LTC Sisti & USA \\
    & & AF Study      & John Munson & \\
    & & STARS Program & Joseph Bats & OSD-STARS \\
\\
    22 April ’85 & Pentagon & Planning of Study\\
\\
    28-29 May ’85 & MITRE, McLean & Navy Software Perspective         & COMO Harry Quest & USN\\
    & & Tactical Flag Com. Ctr.      & CMDR DeMarse     & USN TFCC\\
    & & Discussion re Task Force     & Dr. James P. Wade, Jr. &  OUSDR\&E\\
    & & Army Software Perspective    & BG Alan Salisbury     & USA Sys Command\\
    & & STARS Program                & Joseph Bats           & STARS\\
    & &                              & Dr. Edward Lieblein   & OSD\\
    & & Assessment of STARS Conf.    & Dr. Barry Boehm & \\
    & & SEI/STARS Relationship       & Dr. Mary Shaw& SEI \\
    & & 1982 DoD Joint SW Study      & Larry Druffel & Rational Technology\\
    & & AF Software Perspective      & BG Dennis Brown & USAF\\
\\
    24 June ’85 & Pentagon  & WIS Program & John Gilligan     & Deputy SPO Director\\
 & & & Gene Frank                    & GTE\\
 & & & Don ONeil                     & IBM\\
 & & Executive Session & Don Latham  & ASD(C3I)\\
 & & & Dr. James Wade                & Acting USDR\&E\\
\\
8 July ’85 & MITRE, McLean & Report Discussion & Dr. Danny Cohen & ISI\\
\\
22-23 October ’85 & MITRE, McLean & Stategic Computing Initiative & Dr. Steve Squires & DARPA\\
 & & Strategic Defense Initiative  & Maj. David Audley & SDI\\
 & & Compet. In Contracting Act    & Wayne Wittig      & OASD(A\& L)   \\
 & & STARS                         & Dr. Jack Kramer   & OSD-STARS      \\
\\
26 November ’85 & Pentagon    & DFARS 27.4                & Ms. Pamela Samuelson & SEI           \\
                         & &  SW Test \& Eval. Project  & Dr. Rich DeMilo      & Georgia Tech      \\
\\
22 January ’86 & Pentagon & Report Discussion & Dr. Hicks, Task Force & USDR\&E\\
\\
    4-5 March ’86 & MITRE, McLean & Discussion of DSB Briefing\\
\\
    15 April ’86 & MITRE, Bedford & Report Discussion\\
\\
    27 May ’86 & SEI & Report Discussion\\
\end{tabular}
\end{footnotesize}


\newpage

\addcontentsline{toc}{subsection}{A4. Documents Studied}
\section*{Appendix A4 — Documents Studied and References}

\begin{description}[leftmargin=2in,style=sameline]
    \item [Bailey, Elizabeth, \textit{et al}] An Assessment of the STARS Program, September-October 1985 (IDA Memorandum Report M-137)

    \item [Barbacci, M.R. \textit{et al}] “The Software Engineering Institute: Bridging Practice
        and Potential,” \emph{IEEE Software}, November, 1985

    \item [Boehm, Barry W.] “Understanding and Controlling Software Costs,” \emph{Information Processing 86}, H.J. Kugler, ed., Amsterdam: Elsevier Science Publishers B.V. (North Holland)

    \item [Brocks, Frederick P.] “No Silver Bullet,” \emph{Information Processing 86}, H.J. Kugler, ed., Amsterdam: Elsevier Science Publishers B.V.  (North Holland). Reprinted in \emph{Computer}, April, 1987

    \item [DeMillo, Richard A., \textit{et al}] Software Engineering Environments for Mission-Critical Applications - STARS Alternative Programmatic Approaches (IDA Paper P-1780) August 1984

    \item [DoD]
        DoD Directive 5000.29 Management of Computer Resources in Major
        Defense Systems, 26 April 1976

        Draft DoD Directive 5000.29, 15 January
        1986

        DoD-STD-2167 Defense System Software Development, 4 June 1985

        Draft DoD-STD-2167A Defense System Software Development, 1 April 1987

        DoD Directive 3405.1 Computer Programming Language Policy, 2 April 1987

        DoD Directive 3405.2 Use of Ada in Weapon Systems 30 March 1987

        Strategy for a DoD Software Initiative, 1 October 1982

        Software Technology for Adaptable Reliable Systems (STARS) Program Strategy, 1
        April 1983

        STARS Implementation Approach, 15 March 1983

        48 CFR Parts 214, 215, 227, and 252 Revised Defense Federal Acquisition Regulation
        Supplement Technical Data 10 September 1985

    \item [Druffel, Larry E., \textit{et al}] Report of the DoD Joint Service Task Force
        on Software Problems, 30 July 1982

    \item [Frewin, Gillian, \textit{et al}] “Quality Measurement and Modelling
        - State of the Art Report,” REQUEST Consortium, Esprit Project ESP/800
        Strasbourg, 8 July 1985

    \item [Hamblen, John W.] \textit{Computer Manpower — Supply and Demand by
        States}.  Quad Data Corp., Tallahassee, 1984

    \item [Jones, Victor E., \textit{et al}] Final Report of the Software
        Acquisition and Development Working Group, July 1980

    \item [Jorstad, Norman D., \textit{et al}] Report of the Rights in Data
        Technical Working Group (RDTWG) 23 January 1984 (IDA Record Document
        D-52)

    \item [Lieblein, Edward] “The Department of Defense Software Initiative — A
        Status Report,” \emph{Communications of the ACM, 29}, 8, August, 1986

    \item [Munson, John B., \textit{et al}] Report of the USAF Scientific
        Advisory Board Ad Hoc Committee on the High Cost and Risk of
        Mission-Critical Software December 1983

    \item [NSF] “Projected Response of the Science, Engineering, and Technical
        Labor Market to Defense and Nondefence Needs: 1982-87.” NSF Report
        NSF84-304. 1984

    \item [Parnas, David L.] “Designing Software for Ease of Extension and
        Contraction”, \emph{IEEE Trans on SE, 5}, 2, March 1979, 128-138

    \item [Redwine, Samuel T., \textit{et al}] DoD Related Software Technology
        Requirements Practices, and Prospects for the Future (IDA Paper P-1788)
        June 1984

    \item [Samuelson, Pamela] Toward a Reform of the Defense Department
        Software Acquisition Policy (Working Paper)

    \item [Selin, Ivan, \textit{et al}] Interim Report on Air Force Base Level
        Automation Environment, National Research Council, National Academy
        Press, June 1985

    \item [Taft, William H., IV] Memorandum to the Joint Logistics Commanders,
        12 August 1985

    \item [Vetter, Betty M.] \emph{The Technological Marketplace: Supply and
        Demand for Scientists and Engineers}. Scientific Manpower Commission,
        Washington, 1985

    \item [Vick, Charles R., \textit{et al}] “Methods for Improving Software
        Quality and Life Cycle Cost,” AF Studies Board, National Academy Press,
        1985

    \item [Weinberger, Casper W.] “Remarks Delivered at the Ada Expo 1986”

    \item [Yaru, Nicholas, \textit{et al}] Army Science Board Study on
        Acquiring Army Software, 1983

    \item [Zracket, Charles A., \textit{et al}] “Initiatives to Improve the
        Development of USAF C\textsuperscript{3}I Software,” MITRE, March 1984

\end{description}

\newpage

\addcontentsline{toc}{subsection}{A5. Software — Why Is It Hard?}
\section*{Appendix A5 — Why Is Building Software Hard?}

There are no radical breakthroughs now in view; moreover the very nature of
software makes it unlikely that there will be any – no inventions that will do
for software productivity, reliability, and simplicity what electronics,
transistors, large-scale integration did for computer hardware.\footnote{This
appendix is an extract from “No Silver Bullet”, an invited paper presented in
Dublin by F. P. Brooks at the 1986 Congress of the International Federation of
Information Processing. The full paper is in the Congress Proceedings.} We
cannot expect ever to see two-fold gains every two years.

To see why this is so, and to determine what actions we must follow instead of
hoping for breakthroughs, let us examine the difficulties of the software
development process. We divide them into \emph{essence}, the difficulties
inherent in the nature of the software, and \emph{non-inherent} problems, those
difficulties which today attend its production but which are not inherent.

The non-inherent problems I discuss in the next section. First let us consider
the essence.

The essence of a software entity is a construct of interlocking concepts: data
sets, relationships aunong data items, algorithms, and invocations of
functions. This essence is abstract, in that the conceptual construct is the
same under many different representatiuus.  It is nonetheless highly precise
and richly detailed.

I believe the hard part of building software to be the specification, design,
and testing of this conceptual construct itself, not the labor of representing
it and testing the fidelity of the representation. We still make syntax errors,
to be sure; but they are fuzz compared to the conceptual errors in most
systems.

If this is true, building software will always be hard. There is inherently no
magic.

Let us consider the inherent properties of this irreducible espence of modern
software systems: complexity, conformity, changeability, and invisibility.

\subsection*{Complexity}

Software entities are more complex for their size than perhaps any other human
construct, because no two parts are alike (at least above the statement level).
If they are, we make the two similar parts into one, a subroutine, open or
closed. In this respect software systems differ profoundly from computers,
buildings, or automobiles, where repeated elements abound.

Digital computers are themselves more complex than most things people build;
they have very large numbers of states. This makes conceiving, describing, and
testing them hard. Software systems have orders of magnitude more states than
computers do.

Likewise, a scaling-up of a software entity is not merely a repetition of the
same elements in larger size, it is necessarily an increase in the number of
different elements.  In most cases, the elements interact with each other in
some non-linear fashion, and the complexity of the whole increases much more
than linearly.

Many of the classical problems of developing software products derive from thia
essential complexity and its non-linear increases with size. From the
complexity comes the difficulty of communication among team members, which
leads to product flaws, cost overruns, schedule delays. From the complexity
comes the difficulty of enumerating much less visualizing, all the possible
states of the program, and from that comes the unreliability.  Computer
programs do not break or wear out. The bugs one finds are either design flaws,
implementation errors, or are the consequences of changed environments and
interface.  From the complexity of the functions comes the difficulty of
invoking those functions, which makes programs hard to use. From complexity of
structure comes the difficulty of extending programs to new functions without
creating side effects. From complexity of structure come the unvisualized
states that constitute security trapdoors.

Not only technical problems, but management problems as well come from the
complexity. It makes overview hard, thus impeding conceptual integrity. It
makes it hard to find and control all the loose ends. It creates the tremendous
learning and understanding burden that makes personnel turnover a disaster.

\subsection*{Conformity}

Complexity alone is nothing unique to the software discipline. Physics deals
with terribly complex objects even at the “fundamental” particle level. The
physicist labors on, however, in a firm faith that there are unifying
principles to be found, whether in quarks or in unified field theories.
Einstein repeatedly argued that there must eventually be simplified
explanations of nature, because God is not capricious or arbitrary.

No such faith comforts the software engineer. Much of the complexity he must
master is arbitrary complexity, forced without rhyme or reason by the many
human institutions and systems to which his interfaces must conform. These
differ from interface to interface, and from time to time, not because of
necessity but only because they were designed by different people, rather than
by God.

In many cases the software must conform because it has most recently come to
the scene. In others in must conform because it is perceived as the most
conformable. But in all cases, much complexity cnmes from conformation to other
interfaces; this cannot be simplified out by any redesign of the software
alone.

\subsection*{Changeability}

The software entity is constantly subject to pressures for change. Of course,
so are buildings, cars, computers. But manufactured things are infrequently
changed after manufacture; they are superseded by later models, or essential
changes are incorporated in later-serial-number copies of the same basic
design. Call-backs of automobiles are really quite infrequent; field changes of
computers somewhat less so. Both are much less frequent than modifications to
fielded software.

Partly this is because the software in a system embodies its function, and the
function is the part which most feels the pressures of change. Partly it is
because software can be changed more easily – it is pure thought-stuff,
infinitely malleable. Buildings do in fact get changed, but the high costs of
change, understood by all, serve to dampen the whims of the changers.

All successful software gets changed. Two processes are at work. As a software
product is found to be useful, people try it in new cases at the edge of, or
beyond, the original domain. The pressures for extended function come chiefly
from users who like the basic function amd invent new uses for it.

Successful software also survives beyond the normal life of the machine vehicle
for which it is first written. If not new computers, then at least new disks,
new displays, new printers come along; and the software must be conformed to
its new vehicles of opportunity.

In short, the software product is embedded in a cultural matrix of
applications, users, laws, and machine vehicles. These all change continually,
and their changes inexorably force change upon the software product.

\subsection*{Invisibility}

Software is invisible and unvisualizable. Geometric abstractions are powerful
tools. The floor plan of a building helps both architect and client evaluate
spaces, traffic flows, views. Contradictions become obvious, omissions can be
caught. Scale drawings of mechanical parts and stick-figure models of molecules,
although abstractions, serve the same purpose. A geometric reality is captured
in a geometric abstraction.

The reality of software is not inherently embedded in space. Hence it has no
ready geometric reuresentation in the way that land has maps, silicon chips
have diagrams, computers have connectivity schematics. As soon as we attempt to
diagram software structure, we find it to constitute not one, but several,
general directed graphs, superimposed one upon another. The several graphs may
represent the flow of control, the flow of data, patterns of dependency, time
sequence, name-space relatioiships. These are usually not even planar, much
less hierarchical. Indeed, one of the ways of establishing conceptual control
over such structure is to enforce link cutting until one or more of the graphs
becomes hierarchical [Parnas, 1979].

In spite of progress in restricting and simplifying the structures of software,
they remain inherently unvisualizable, thus depriving the mind of some of its
most powerful conceptual tools. This lack not only impedes the process of
design within one mind, it severely hinders communication among minds.

\subsection*{Past Breakthroughs Solved Accidental Difficulties}

If we examine the three steps in software technology that have been most
fruitful in the past, we discover that each attacked a different major
difficulty in building software, but they have been the non-inherent, not the
essential, difficulties. We can also see the natural limits of each such
approach.

\subsubsection*{High-Level Languages}

Surely the most powerful stroke for software productivity, reliability, and
simplicity has been the progressive use of high-level languages for
programming. Most observers credit that development with at least a factor of
five in productivity, and with concomitant gains in reliability, simplicity,
and comprehensibility.

What does a high-level language accomplish? It frees a program from much of its
incidental complexity. An abstract program consists of conceptual constructs:
operations, data-types, sequences, and cornmmunication. The concrete machine
program is concerned with bits, registers, conditions, branches, channels,
disks, and such. To the extent that the high-level language embodies the
constructs one wants in the abstract program and avoids all lower ones, it
eliminates a whole level of complexity that was never inherent in the program
at all.

The most a high-level language can do is to furnish all the constructs the
programmer imagines in the abstract program. To be sure, the level of our
sophistication in thinking about data structures, data types, and operations is
steadily rising, but at an ever-decreasing rate. And language development
approaches closer and closer to the sophistication of users.

Moreover, at some point the elaboration of a high-level language becomes a
burden that increases, not reduces, the intellectual task of the user who
rarely uses thc esoteric constructs.

\subsubsection*{Time-Sharing}

Most observers credit time-sharing with a major improvement in the productivity
of programmers and in the quality of their product, although not so large as
that brought by high-level languages.

Time-sharing attacks a quite different difficulty. Time-sharing preserves
immediacy, and hence enables one to maintain an overview of complexity. The
slow turnaround of batch programming means that one inevitably forgets the
details, if not the very thrust, of what he was thinking when he stopped
programming and called for compilation and execution. This interruption of
consciousness is costly in time, for one must refresh. The most serious effect
may well be the decay of grasp of all that is going on in a complex system.

Slow turn-around, like machine-language complexities, is an unnecessary rather
than an essential difficulty of the software process. The limits of the
contribution of time-sharing derive directly. The principal effect is to
shorten system response time. As it goes to zero, at some point it passes the
human threshold of noticeability, about 100 milliseconds.  Beyond that no
benefits are to be expected.

\subsubsection*{Unified Programming Environments}

Unix and Interlisp, the first integrated programming environments to come into
widespread use, are perceived to have improved productivity by integral
factors. Why?  They attack the incidental difficulties of using programs
together, by providing integrated libraries, unified file formats, and pipes
and filters. As a result, conceptual structures that in principle could always
call, feed, and use one another can indeed easily do so in practice.

This breakthrough in turn stimulated the development of whole toolbenches,
since each new tool could be applied to any programs using the standard
formats. How much more gain can be expected from the exploding researches into
better programming environments?  One’s instinctive reaction is that the
big-payoff problems were the first attacked, and have been solved: hierarchical
file systems, uniform file formats so as to have uniform program interfaces,
and generalized tools. Language-specific smart editors are developments not yet
widely used in practice, but the most they promise is freedom from syntactic
errors and simple semantic errors.

Perhaps the biggest gain yet to be realized in the programming environment is
the use of integrated database systems to keep track of the myriads of details
that must be recalled accurately by the individual programmer and kept current
in a group of collaborators on a single system.

Surely this work is worthwhile, and surely it will bear some fruit in both
productivity and reliability. But by its very nature, the return from now on
must be marginal.

\subsection*{Conclusion}

All of the technological attacks of the software process are fundamentally
limited by the productivity equation:

\[\textit{time of task} = \sum_i (\textrm{frequency})_i \times (\textrm{time})_i\]

If, as I believe, the conceptual components of the task are now taking most of
the time, then no amount of activity on the task components that are merely the
expression of the concepts can give large productivity gains.

We are left with the inherent task — getting the complex concepts right, and
changing them correctly as the world keeps changing about them. This is a human
activity, and a labor-intensive one.

\newpage

\addcontentsline{toc}{subsection}{A6. Proposal for a new “Rights in Software” Clause}
\section*{Appendix A6 — Proposal for a new “Rights in Software” Clause}

\textit{Porting note —} Interested readers are referred to the original PDF of
this
report\footnote{\url{https://apps.dtic.mil/dtic/tr/fulltext/u2/a188561.pdf}},
or a direct search for Technical Report SEI-86-TR-2 of the Software Engineering
Institute of Carnegie Mellon University, authored by Pamela Samuelson.

But I didn’t feel like porting over the entire report.

\newpage

\addcontentsline{toc}{subsection}{A7. Proposal for a Module Market}
\section*{Appendix A7 — A Proposal for an Ada Software Module Market}

An Ada Software Module Market enterprise could perhaps operate and become
viable on the following basis:

\begin{enumerate}
    \item The ASOMM would be established for the purpose of supporting Ada
        through disseminating Ada modules. It would also accept Ada support
        tools written in other languages.
    \item The ASOMM would set standards for module catalog description that
        specify precisely the portability properties (e.g., “compiles and
        operates in Unix 4.2 bsd environments, including these which we have
        tested: SUN, VAX11”). It might also set standards for other description
        attributes – accuracy, speed, function. It would set standards for the
        form of source code and documentation and of test cases and test
        drivers.
    \item The owners of modules would set the prices for their offerings, but
        ASOMM would have a uniform set of terms and conditions, so that
        prospective users would have minimum paperwork.
    \item ASOMM would handle all marketing, distribution, and licensing,
        charging a substantial (but perhaps sliding) commission on revenues.
    \item ASOMM would not itself undertake module development, enhancement,
        documentation, repair, support, validation, certification, or warranty.
        It would be a mail-order software dealer.
    \item All modules would include full copyrighted source, except perhaps for
        security submodules, which might be object-only.
    \item The standard terms and conditions would include at least four kinds
        of licenses, at different prices:
\begin{itemize}
        \item One copy, limited-period trial use, price refundable except for
            rental fee.
        \item Per machine/copy, unlimited use
        \item Site linceses \emph{[sic]}, at least for personal computers and
            workstations
        \item Per incorporation as a component in a larger product, with free
            sub-licensing rights. This would enable an incorporator to do a
            one-time transaction and not undertake a long-run paperwork burden.
\end{itemize}
    \item The standard terms and conditions would include different levels of
        support for licensed users (but not sub-licensees):
\begin{itemize}
        \item Fully supported by the owner, perhaps for an annual fee.
        \item Supported by the owner on a fixed-fee-per-fix basis.
        \item Support negotiable with the owner.
        \item Unsupported. Caveat emptor.
\end{itemize}
    \item ASOMM would maintain lists of licensed users for recall and update
        purposes.
    \item For a modest surcharge, a user could get, along with the module, a
        list of the other licensed users willing to be listed.
\end{enumerate}

\end{document}
